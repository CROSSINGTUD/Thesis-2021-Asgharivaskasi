There are several tools that detect misuses of Java cryptographic APIs within programs that either employ static or dynamic analysis or a combination of the two. These tools may use different approaches or improve upon an old one. Among those tools, \cognicryptsast{} and \codyze{} seem to follow a similar approach. Both tools use a white-listing approach and define rules to specify the correct usage of cryptographic APIs using DSLs, \crysl{} in \cognicryptsast, and \MARK{} in \codyze. This thesis aims to provide a fair comparison of \codyze{} and \cognicryptsast{} to discover the advantages and disadvantages of each approach to be able to improve them and create better analyzers in the future. Therefore, we conducted a theoretical and practical comparison of the two tools and their DSLs by reviewing their documentation, related papers, and source codes. Further, we translated the rules from one language to another to compare the DSLs' expressiveness. In addition, we evaluated the performance (precision and recall) of the tools by analyzing two benchmarks.

The rule translation revealed that not all elements of one language were translatable to the other. Moreover, we determined some analysis properties of each tool based on the results of the theoretical comparison and the performance evaluation. According to our results, \codyze{} and \cognicryptsast{} performed similarly; however, some aspects of both tools and their respective DSLs could be improved for tools to perform more effectively and efficiently.
