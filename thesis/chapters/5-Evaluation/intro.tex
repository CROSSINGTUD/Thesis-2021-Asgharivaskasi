\chapter{Evaluation}
\label{ch:eval}
In this chapter, we will evaluate and compare the performance (precision and recall) of \codyze{} and \cognicryptsast{} Eclipse plugins. To accomplish this, we will use Java code samples that contain the correct and incorrect usages of cryptographic APIs and compare the analysis results each tool generates after analyzing the code samples.

Consequently, we require a benchmark for comparing the efficiency of the tools in detecting cryptographic vulnerabilities. Unfortunately, \codyze{} and \cognicryptsast{} do not provide any generally accepted benchmark in this domain; therefore, we searched for alternative resources. One of the benchmarks we can use to compare analysis results that is more recent and relevant is the \cryptoapibench{} \cite{cryptoapibench}, which is the benchmark for \cryptoguard{} \cite{sr19} and covers a wide range of misuse instances. We will discuss \cryptoapibench{} and the results of analyzing it using both tools in section \ref{sec:apibench}. 

Since \cryptoapibench{} might be biased towards \cryptoguard{} and therefore may not provide us with an accurate comparison, we further used \cognicrypttestgen{}\cite{rakshit}. \cognicrypttestgen{} is a test generator for cryptographic APIs. It generates test suites that contain test cases with all correct and incorrect usages of an API. It consumes \crysl{} rules as an input, parses them, and produces test suites for each rule. We will explain more about the test generation and the result of analyses in the Section \ref{sec:testgen}.

To avoid confusion, we refer to the translated rules as follows.
The translated Bouncy Castle JCA \crysl{} rules to \MARK, the translated Bouncy Castle JCA \MARK{} rules. The translated JCA \crysl{} rules to \MARK, the translated JCA \MARK{} rules. The translated Bouncy Castle \MARK{} rules to \crysl, the translate Bouncy Castle \crysl{} rules.

The following setup section describes the environment in which we conducted the evaluations. Next, we discuss the results of analyzing \cryptoapibench{} using \cognicryptsast{} and \codyze, followed by discussing the results of analyzing \cognicrypttestgen{} generated tests.

\subsubsection*{Setup}
To analyze benchmarks with \codyze{} and \cognicryptsast, we employed the same virtual machines with analyzers installed in them as described in the setup of Section \ref{sec:practical}. There is a total of 6 GB of memory allocated to \codyze's language server as \codyze{} developers recommend using 6 GB or more. Since our virtual machine has a maximum memory of 8 GB, we dedicated 6 GB to the language server. The type state analysis is set to \code{within functions only}. The other option, \code{whole program}, does not work, therefore the \codyze{} developer recommended setting it to \code{within functions only}. The call graph construction algorithm of \cognicryptsast{} is set to CHA, which is the default. In this chapter, we used the original \crysl{} and \MARK{} rules as described in the Section \ref{sec:rulesetup} and the translated rules that are described in Section \ref{sec:practicdsl}.


\section{Evaluation with \cryptoapibench}
\cryptoapibench{} is the first benchmark that we utilized to evaluate \cognicryptsast and \codyze. In this section, we will first discuss the \cryptoapibench{} and the types of misuses it covers. Following the discussion of experiment execution, we examine the results of the analyses of \cryptoapibench{} using \codyze{} and \cognicryptsast. 

According to Afrose \etal  \cite{cryptoapibench}, \cryptoapibench{} provides 171 cases for assessing the quality of various cryptographic vulnerability detection tools. \cryptoapibench{} addresses 16 different types of cryptographic and SSL/TLS API misuse vulnerabilities. These include hardcoded secrets, improper certificate validations, improper hostname validations, insecure symmetric and asymmetric cryptographic primitives, and insecure hash functions \cite{cryptoapibench}. The \cryptoapibench{} consists of 40 basic tests, 131 advanced tests, which includes 40 inter-procedural, 19 field-sensitive, 20 combined tests (combination of inter-procedural and field-sensitive), and 20 path-sensitive test cases \cite{cryptoapibench}.

Each of these 16 categories of misuses of cryptographic APIs is regarded by the \cryptoapibench{} as a cryptographic threat model, and each threat model has a corresponding cryptographic API \cite{cryptoapibench}. The \cryptoapibench{} provides test cases for which neither \codyze{} nor \cognicryptsast{} provides rules for the corresponding APIs, namely, javax.net.ssl.HostnameVerifier, javax.net.ssl.X509TrustManager, javax.net.ssl.SSLSocket, and java.net.URL. Therefore, we do not consider the threat models related to those APIs in our evaluation because both tools did not identify any misuses in those models. We will only use the remaining 12 threat models that are shown in the Table \ref{tab:cryptobench}. The following is a brief explanation of these 12 threat models presented by \cryptoapibench{} in order to gain an understanding of the type of vulnerability they introduce.

\emph{Cryptographic Key}. When using a key generated by the javax.crypto.spec.SecretKeySpec API for encryption, the Byte array that SecretKeySpec takes as input should be unpredictable and not constant or hard-coded. Otherwise, the attacker could easily read the key to obtain sensitive information. 

\emph{Passwords in PBE (Password-based Encryption)}. It relates to the use of a constant or hardcoded password in the javax.crypto.spec.PBEKeySpec API to generate a SecretKey. A hardcoded or constant password may allow an attacker to acquire it and gain access to the key.

\emph{Passwords in KeyStore}. The java.security.KeyStore API is used to store cryptographic keys or certificates. The KeyStore requires a password to access the stored keys, which should not be hardcoded or a constant. Otherwise, the keys and certificates stored in the KeyStore could be accessed by unauthorized parties.

\emph{Pseudorandom Number Generator (PRNG)}. The random number generator algorithm used by java.util.Random (Knuth's subtractive \cite{sok}) is proven to be insecure, while java.security.SecureRandom generates a random number that is non-deterministic and unpredictable. Therefore, it is recommended to use SecureRandom instead of Random because the latter does not produce a completely random number.

\emph{Seeds in Pseudorandom Number Generator (PRNG)}. When using SecureRandom to generate random numbers, we may encounter the same random number in every run if we use a constant or not randomly generated seed as SecureRandom's parameter.

\emph{Salts in Password-based encryption (PBE)}. Javax.crypto.spec.PBEParameterSpec API takes the salt and iteration count when setting parameters for Password-based encryption. Salt should not be hardcoded, or a constant but rather should be randomly generated. Otherwise, it may result in producing an insecure key.
 
\emph{Mode of Operation}. ECB (Electronic Codebook) is not considered a secure mode of operation in javax.crypto.Cipher since it does not conceal data patterns well and may disclose information about the plaintext. CBC or Galois/Counter (GCM) Mode should be used instead.

\emph{Initialization Vector (IV)}. To enhance cryptography security, initialization vectors (IVs) are used during encryption and decryption. A constant/static initialization vector may introduce vulnerabilities. Therefore, it is recommended to use a random initialization vector in the crypto.spec.IvParameterSpec API.

\emph{Iteration Count in Password-based Encryption (PBE)}. For Password-based Encryption (PBE) with the javax.crypto.spec.PBEParameterSpec API, it requires salt and iteration count. The Public Key Cryptography Standards (PKCS), which is a set of public-key cryptography standards published by RSA Security LLC (an American company that focuses on encryption and encryption standards \cite{rsa}) #5 version 2.0 \cite{pbeiteration}, recommends that the number of iterations should be greater than 1000 to provide adequate security. Therefore, iteration counts less than 1000 are considered insecure. 

\emph{Symmetric Ciphers}. Symmetric cryptography employs the same key for both encryption and decryption. AES is the preferred symmetric cipher, as several other symmetric ciphers, including DES, Blowfish, RC4, RC2, IDEA, and RC4, are considered broken.

\emph{Asymmetric Ciphers}. In asymmetric cryptography, a pair of keys, including a public key and a private key, is used to encrypt and decrypt data. It is, however, recommended to use 2048 bit key size for some asymmetric ciphers, such as RSA, because they are considered broken with 1024 bit key size.

\emph{Cryptographic Hash Functions}. In cryptography, hash functions convert an arbitrary message to a fixed-size value known as a hash or message digest, which is utilized to verify message integrity, digital signature, and authentication. When two inputs of the same cryptographic hash function produce the same hash value, it is considered to be broken. SHA-256 is an alternative to insecure hash functions such as SHA1, MD4, MD5, and MD2.


\subsection{Description of the execution of the experiment}

To collect analysis results, we analyzed the \cryptoapibench{} source code with \cognicryptsast{} Eclipse plugin by triggering the start button and analyzing the entire project, and with \codyze{} Eclipse plugin by opening all the class files in the Eclipse editor. As JCA is the most commonly used cryptographic library for Java \cite{snb16}, we chose to use the JCA rulesets. However, the \MARK{} developers did not provide rules for the JCA API, so we used \crysl{} rules for the JCA API \cite{apirules} in \cogcnicryptsast{} and \MARK{} rules for Bouncy Castle API \cite{codyzegit} in \codyze.

The Misuses that \codyze{} finds in a program in Eclipse are not displayed with a red flag except for the misuses related to the forbidden methods. They are instead displayed as information, such as when a \MARK{} rule is verified. This made it difficult to distinguish between violations and verifications of \codyze's findings in Eclipse and increased the risk of mistakes occurring when collecting the analyses results. In our meeting with a \codyze{} developer, they mentioned that the problem lies within the file \code{CpgDocumentService.java}\footnote{\url{https://github.com/Fraunhofer-AISEC/codyze/blob/main/src/main/java/de/fraunhofer/aisec/codyze/crymlin/connectors/lsp/CpgDocumentService.java}} . After \codyze{} has gathered all findings in a program, it categorizes their severity level into three categories: information, warning, and error. Nevertheless, it does not identify misuses within the error category; instead, misuses are considered information, which is the default severity level.

After performing the analysis, we obtained the true positives and false positives based on the results logs of each tool. Since our goal is to compare the cryptographic vulnerability detection of the two tools, we considered only cryptographic misuse alerts and disregard unrelated information. When a tool generates an alert based on the correct reason in a vulnerable test case, it is considered a true positive (TP). If in a correct test case of a threat model, a tool identifies a misuse related to that threat model, then it is considered as a false positive (FP).


% true positive is when the analysis identified a misuse that exists, and true negative is when the analysis does not indicate a misuse where one did not occur.


\subsection{Examining the results of analyses}

In the following, we describe the results of evaluating the \cryptoapibench{} with each detection tool and compare their performances. Table \ref{tab:cryptobench} presents the number of true positives (TP) and false positives (FP) detections of vulnerabilities by the \codyze{} and \cognicryptsast{} in the 12 cryptographic threat models in \cryptoapibench{} that we previously discussed. In this context, true positive is when a tool detects misuse in a test case, and it is a vulnerability based on the CryptoAPI-Bench\_details.xlsx file available on \cryptoapibench's Github page \cite{benchgithub}. This Excel file provides an overview of secure and non-secure code as well as the vulnerabilities. Further, a false positive is when a tool identifies a misuse in a test case with no vulnerability based on the Excel file. TTP stands for total true positives, the number of test cases with vulnerabilities provided by the \cryptoapibench. TTN stands for total true negatives, the number of test cases provided by \cryptoapibench{} with the correct usages of cryptographic APIs. TTN and TTP are calculated based on the Excel file. The last two columns in Table \ref{tab:cryptobench}, indicate the recall and precision of \codyze{} and \cognicryptsast{} when analyzing each threat model. Pre and Rec are acronyms for precision and recall, respectively.

\newcolumntype{P}[1]{>{\centering\arraybackslash}p{#1}}
\newcommand\mc[1]{\multicolumn{1}{c}{#1}}
\newcommand\mC[1]{\multicolumn{1}{C}{#1}}
\begin{table}[hp]
 \begin{adjustwidth}{-1em}{} 
%  \captionsetup{justification=centering}
\setlength\tabcolsep{3.3pt} 
\small
\centering
\begin{tabular}{|c|l|c|c|c|c|}
 \hline
 No.& Threat Models & TTP & TTN & 
 \begin{tabular}{cccc}
    \multicolumn{4}{c}{\codyze{}}\\
    TP&FP&{\hskip 0.2in}Pre(\%)&Rec(\%)\\
\end{tabular}
&
\begin{tabular}{cccc}
    \multicolumn{4}{c}{\cognicryptsast{}}\\
    TP&FP&Pre(\%)&Rec(\%)\\
\end{tabular} \\

 \hline
 \hline
1& Cryptographic Key& 8 & 2 & 0 {\hskip 0.15in} 0 {\hskip 0.5in} 0{\hskip 0.4in} 0 & 8 {\hskip 0.2in} 2 {\hskip 0.2in} 80,00 {\hskip 0.2in} 100\\
2& Password in PBE & 8 & 3&0 {\hskip 0.15in} 0 {\hskip 0.5in} 0{\hskip 0.4in} 0 & {\hskip 0.1in}6 {\hskip 0.2in} 5 {\hskip 0.2in} 54,54 {\hskip 0.2in} 75,00\\
3&Password in KeyStore & 7& 3& 0 {\hskip 0.15in} 0 {\hskip 0.5in}  0{\hskip 0.4in} 0 & {\hskip 0.1in}6 {\hskip 0.2in} 2 {\hskip 0.2in} 75,00 {\hskip 0.2in} 85,71\\
4& PRNG  & 1& 1& 0{\hskip 0.2in} 0 {\hskip 0.5in} 0 {\hskip 0.35in} 0{\hskip 0.065in}& 0{\hskip 0.25in} 0 {\hskip 0.45in} 0 {\hskip 0.3in} 0{\hskip 0.3in}\\
5& Seed in PRNG  & 14&3 &0 {\hskip 0.15in} 0 {\hskip 0.5in} 0 {\hskip 0.35in} 0 & 0{\hskip 0.25in} 0 {\hskip 0.45in} 0 {\hskip 0.3in} 0{\hskip 0.3in}\\
6& Salt in PBE  & 7&2 & 0 {\hskip 0.15in} 0 {\hskip 0.5in}  0{\hskip 0.4in} 0 & 7 {\hskip 0.2in} 2 {\hskip 0.2in} 77,77 {\hskip 0.2in} 100\\
7& Mode of Operation  & 6&2 &{\hskip 0.05in}1 {\hskip 0.2in} 1 {\hskip 0.3in} 50,00{\hskip 0.15in} 16,66 &{\hskip 0.1in}5 {\hskip 0.2in} 1 {\hskip 0.2in} 83,33 {\hskip 0.2in} 83,33\\
8& Initialization Vector  & 8&2& 8 {\hskip 0.2in} 2 {\hskip 0.3in} 80,00{\hskip 0.2in} 100 & 8 {\hskip 0.2in} 0 {\hskip 0.3in} 100 {\hskip 0.2in} 100\\
9& Iteration Count in PBE & 7&2& 0 {\hskip 0.2in} 0 {\hskip 0.5in} 0{\hskip 0.4in} 0 &{\hskip 0.1in} 5 {\hskip 0.2in} 2 {\hskip 0.2in} 71,42 {\hskip 0.2in} 71,42\\
10& Symmetric Cipher& 30&6& 30 {\hskip 0.15in} 6 {\hskip 0.3in} 83,33{\hskip 0.2in} 100 & {\hskip 0.1in}20 {\hskip 0.2in} 5 {\hskip 0.2in} 80,00 {\hskip 0.2in} 66,66\\
11& Asymmetric Ciphers & 5& 1& 0 {\hskip 0.2in} 0 {\hskip 0.5in} 0{\hskip 0.4in} 0 & {\hskip 0.1in}4 {\hskip 0.25in} 1 {\hskip 0.2in} 80,00 {\hskip 0.2in} 80,00\\
12& Cryptographic Hash  & 24&5& 24 {\hskip 0.15in} 4 {\hskip 0.3in} 85,71{\hskip 0.2in} 100 & {\hskip 0.1in}20 {\hskip 0.2in} 4 {\hskip 0.2in} 83,33 {\hskip 0.2in} 83,33\\

 \hline
 \multicolumn{2}{|l|}{\textbf{Total}} & 125 & 32& {\hskip 0.1in} 63 {\hskip 0.1in} 13 {\hskip 0.3in} \textbf{82,89} {\hskip 0.1in} \textbf{50,80} & {\hskip 0.1in} 89 {\hskip 0.15in} 24 {\hskip 0.2in} \textbf{78,76} {\hskip 0.1in} \textbf{71,20} {\hskip 0.3in}\\
%  \hline
%  \multicolumn{2}{|l|}{\multirow{2}{*}{ \textbf{Results}}} & \multicolumn{2}{|l|}{ \textbf{Precision(\%)}} &  \textbf{82,89} & \textbf{78,76} &-&-\\ 
%  \multicolumn{2}{|l|}{} & \multicolumn{2}{|l|}{ \textbf{Recall(\%)}} &  \textbf{50,80}&  \textbf{71,20}& -&-\\ 
 \hline

\end{tabular}
\caption{\label{tab:cryptobench} Comparison of \codyze{} and \cognicryptsast{} analysis on 12 threat models in \cryptoapibench{}'s 171 test cases.}
\end{adjustwidth}
\end{table}

Based on the Table \ref{tab:cryptobench}, there are 10 models of cryptographic threads that are covered by \cognicryptsast, while \codyze{} covers 4 models. From 125 vulnerable test cases, \codyze{} identified 63 and \cognicryptsast{} identified 89 true positives. \codyze{} and \cognicryptsast{} generate a total of 13 and 24 false alarms (false positives), respectively. In test cases related to the SecureRandom API (threat models 4 and 5 in the Table \ref{tab:cryptobench}), \cognicryptsast{} does not detect any misuses, despite the fact that this type of misuse is explicitly specified by the SecureRandom \crysl{} rule of JCA APIs. This is clearly a false negative. \codyze{} does not detect any misuses for the models 1 to 6 and 9 and 11. It is because there are no \MARK{} rules to specify those vulnerabilities. 

As we mentioned before in Section \ref{sec:codyze}, \MARK{} developers wrote \MARK{} rules based on BSI \cite{BSI} TR-02102-1 version 2019-01. The 2019-01 version of BSI TR-02102-1 is not publicly available; therefore, we contacted the BSI and asked for that version to verify that all the specifications of cryptographic APIs described therein are implemented as \MARK{} rules. We received 2019-01 version \cite{bsi19} very late in the thesis timeline, and unfortunately, it was in German. We requested the English version but did not receive it when this thesis was written. However, we were able to check the rules with the help of translation tools and the latest version of BSI TR-02102-1 (version 2021-01) \cite{bsiTR}.  In some cases, the rules in the BSI could not be described as \MARK{} rules because \MARK{} maintainers could not interpret them in \MARK{} or Bouncy Castle (the API for which the \MARK{} rules are written) did not provide the implementation for that rule.

For example, in section 2.1.2 (Betriebsbedingungen) of the BSI guideline, it is stated that the initialization vectors must not be repeated during the lifetime of a key, i.e., they should not be used for two different ciphers at the same time. Because it requires sufficient knowledge about the program's dynamic behavior, the \MARK{} developers were unable to define it as a \MARK{} rule; however, it is not clear whether this rule is included in the analysis. Another example is section 3.5 (RSA), which is related to the RSA Cipher and indicates that the size of the modulus in calculating the key length of the RSA Cipher must be at least 2000. It cannot be defined in \MARK{} since we cannot adequately reason about the modulus of the RSA key.


The Table \ref{tab:cryptobench} indicates that \codyze{} has a higher precision than \cognicryptsast{}, which means that it can identify misuses that \cognicryptsast{} cannot detect. Meanwhile, \cognicryptsast{} has a better recall than \codyze, which indicates that \cognicryptsast{} has identified more misuses than \codyze, likely because \codyze{} does not cover all threat models as it lacks the \MARK{} rules that specify cryptographic misuses in the missing models. 

The only case in which \codyze{} could not detect all misuses was in the mode of operation (number 7 in the Table \ref{tab:cryptobench}) model. In only one of the test cases, \codyze{} was able to identify an incorrect mode of Cipher. In the other test cases involving such misuse, \codyze{} detected an invalid Cipher rather than an invalid mode of Cipher, which is a false positive.

According to the \crysl{} rule for the class PBEKeySpec of JCA API (see Listing \ref{lst:orgpbecryslRule}), the iteration count should be more than 10000; whereas, based on \cryptoapibench, iteration counts higher than 1000 are acceptable. One of the false positives in the \cognicryptsast{} results for the ninth threat model was caused by this difference.

It should also be noted that \codyze{} and \cognicryptsast{} produce false positives for misuses that are not listed in the threat model of \cryptoapibench. For instance, \cognicryptsast{} identifies a misuse that indicates that the key used by Cipher is not generated correctly as a generatedKey. It occurred in every test case containing a Cipher object with a Key parameter. This is a true positive for the cases where the key is generated with an improper algorithm or key size, but for the cases where a key is properly generated based on the \crysl{} rules, this is a false positive. Nine of the test cases contain this false positive. This issue has already been reported to \cognicryptsast's Github page\footnote{https://github.com/eclipse-cognicrypt/CogniCrypt/issues/457}. 

In addition to the misuses specified by \cryptoapibench{}, \codyze{} and \cognicryptsast{} also detect other misuses in the benchmark. If the provider of an API is not specified, \codyze{} throws a provider error. According to the \MARK{} rules, the provider should always be specified so that APIs used in the code and the API of \MARK{} ruleset used to analyze, have the same provider such as the Bouncy Castle provider. For example, in the mode of operation threat model, \codyze{} detected that there was no provider specified when using the Cipher and KeyGenerator APIs. All the provider misuses discovered by \codyze{} were true positives. Additionally, there were errors detected by \codyze{} concerning calls to java.security.SecureRandom that were forbidden. The findings were also true positives according to the \MARK{} rule for the SecureRandom from the Bouncy Castle API. The forbidden calls are shown in Lines \ref{line:forbidpbe1} and \ref{line:forbidpbe2} of Listing \ref{lst:orgsecureRandomMARK}.
Another finding by \codyze{} was that padding was not applied to the RSA cipher. Based on the \MARK{} rules for Cipher, RSA Cipher must always have a padding (proper paddings are specified in the rule). Therefore \codyze{} found such misuses and they are true positives. According to Naccache \etal \cite{rsapadding}, it is necessary to apply encryption padding in order to prevent dictionary attacks. A dictionary attack is a type of brute force attack in which a hacker uses a list of common words and phrases to attempt to crack a password-protected security system. Furthermore, there was an error related to the Cipher order, as the order in which the Cipher methods were called did not comply with the \MARK{} order rule for Cipher (true positive). The same error was generated more than once within the same test case, which was redundant and unnecessary. 

\cognicryptsast{} detects misuses that are not included in the \cryptoapibench's benchmark, namely invalid ordering of PBEKeySpec, MessageDigest, and Cipher.
Taking the \crysl{} rules for the JCA API into consideration, the order of Cipher error is a false positive, whereas the order of PBEKeySpec is the result of the absence of a call to the clearPassword method as the final call and is a true positive. The MessageDigest invalid order is also a true positive based on the \crysl{} rule.


Overall, \codyze's precision in detecting misuses is approximately 4 percent better than \cognicryptsast. However, when we consider the four threat models for which both tools had rules (threat models 7, 8, 10, and 12), we can observe that \cognicryptsast{} has a better accuracy for two of the models and \codyze{} has a better accuracy for the other two. Further, the difference between precisions is greater when \cognicryptsast's precision is higher (7 and 8), since \codyze{} usually produces more false positives than \cognicryptsast. Regarding the recall, \codyze{} does not provide sufficient \MARK{} rules to define cryptographic vulnerabilities of all threat models, and as a result, its recall value is 20 percent lower than that of \cognicryptsast. 




The next table (Table \ref{tab:cryptobenchadvanced}) shows the result of analyzing \cryptoapibench{} advanced test cases that both \codyze{} and \cognicryptsast{} have rules for the cryptographic APIs of their threat models. This includes the mode of operation (javax.crypto.Cipher), initialization vector (javax.crypto.spec.IvParameterSpec), symmetric cipher (javax.crypto.Cipher) and cryptographic Hash (java.security.MessageDigest). They are categorized into 7 categories that we shortly describe below.

In two-interprocedural cases, the source of vulnerability in one procedure is passed as an argument to the parameter of another procedure and then used in the crytographic API.
In three-interprocedural cases, the source of vulnerability is passed as an argument to another procedure and then passed again to the third procedure. These inter-procedural test cases measure the ability to handle the data flow. 

The field-sensitive cases check the ability of the tools in performing field-sensitive data flow analysis. Combined cases contain test cases with a combination of inter-proceural and field sensitivity. Path-sensitive test cases include conditional branches to examine the accuracy of determining the source of a vulnerability. Miscellaneous test cases are intended to assess the tool's ability to detect irrelevant constraints and other interfaces, such as Map. In multiple class test cases, the source of vulnerabilities that is originated in one class is passed to another class and used in a cryptographic API.


\newcolumntype{P}[1]{>{\centering\arraybackslash}p{#1}}
\begin{table}[H]
 \begin{adjustwidth}{-1em}{} 
\centering
\setlength\tabcolsep{3pt}
\small
\begin{tabular}{ |p{4cm}|P{1cm}|P{1cm}|P{5cm}|P{5cm}|}

 \hline
 Advanced Test Cases & TTP & TTN & \begin{tabular}{cccc}
    \multicolumn{4}{c}{\codyze{}}\\
    TP&FP&Pre(\%)&Rec(\%)\\
\end{tabular}&
\begin{tabular}{cccc}
    \multicolumn{4}{c}{\cognicryptsast{}}\\
    TP&FP&Pre(\%)&Rec(\%)\\
\end{tabular} \\

 \hline
 \hline
Two-Interprocedural&  11 & 0 &10 {\hskip 0.2in} 0 {\hskip 0.2in} 100{\hskip 0.2in} 90,90 & 11 {\hskip 0.2in} 0 {\hskip 0.2in} 100 {\hskip 0.2in} 100\\
Three-Interprocedural& 11  & 0& 10 {\hskip 0.2in} 0 {\hskip 0.2in} 100{\hskip 0.2in} 90,90 & 11 {\hskip 0.2in} 0 {\hskip 0.2in} 100 {\hskip 0.2in} 100\\
Field Sensitive& 11& 0& 10 {\hskip 0.2in} 0 {\hskip 0.2in} 100{\hskip 0.2in} 90,90 & {\hskip 0.1in}2 {\hskip 0.25in} 0 {\hskip 0.2in} 100 {\hskip 0.2in} 18,18\\
Combined Case& 11& 0&10 {\hskip 0.2in} 0 {\hskip 0.2in} 100{\hskip 0.2in} 90,90 & {\hskip 0.1in}1 {\hskip 0.25in} 0 {\hskip 0.2in} 100 {\hskip 0.25in} 9,09\\
Path Sensitive& 0&11 &0 {\hskip 0.2in} 11 {\hskip 0.3in} 0 {\hskip 0.4in} 0 {\hskip 0.2in}& 0 {\hskip 0.2in} 10 {\hskip 0.3in} 0 {\hskip 0.3in} 0{\hskip 0.4in}\\
Miscellaneous Cases& 2&0 &  2 {\hskip 0.3in} 0 {\hskip 0.2in} 100{\hskip 0.3in} 100{\hskip 0.4in} & 2 {\hskip 0.25in} 0 {\hskip 0.2in} 100 {\hskip 0.2in} 100\\
Multiple Class methods& 11&0 &10 {\hskip 0.2in} 0 {\hskip 0.2in} 100{\hskip 0.2in} 90,90 & 11 {\hskip 0.2in} 0 {\hskip 0.2in} 100 {\hskip 0.2in} 100\\

 \hline
 \textbf{Total} & 57 & 11& {\hskip 0.1in}52 {\hskip 0.2in} 11 {\hskip 0.15in} \textbf{82,53}{\hskip 0.1in} \textbf{91,22}{\hskip 0.1in} & {\hskip 0.15in}38 {\hskip 0.15in} 10 {\hskip 0.1in} \textbf{79,16} {\hskip 0.1in}\textbf{66,66}\\
 \hline
%  \multirow{2}{*}{ \textbf{Results}} & \multicolumn{2}{|l|}{ \textbf{Precision(\%)}} &  \textbf{82,53} & \textbf{79,16}&-&-\\ 
%   & \multicolumn{2}{|l|}{ \textbf{Recall(\%)}} &  \textbf{91,22}&  \textbf{66,66}&-&-\\

\end{tabular}
\caption{\label{tab:cryptobenchadvanced} Comparison of \codyze{} and \cognicryptsast{} analyses of \cryptoapibench's 68 advanced test cases (of all 12 threat models in Table \ref{tab:cryptobench}) that both tools have common corresponding cryptographic rules for them.}
\end{adjustwidth}
\end{table}

Table \ref{tab:cryptobenchadvanced} indicates that when we restrict the test cases to those for which both tools have rules that relate to them and are capable of analyzing them, then \codyze's recall increases to 91,22 percent. The only vulnerability that \codyze{} could not identify in the test cases is related to the mode of operation, as previously explained. If we examine the cases individually, \cognicryptsast{} provides better recall than \codyze, except in the field-sensitive and combined cases, in which \cognicryptsast{} detected only three misuses in 22 tests of field-sensitive and combined cases, whereas \codyze{} detected all of them. Therefore, we can conclude that \codyze{} is field-sensitive, and \cognicryptsast{} is partially field-sensitive. However, in Section \ref{sec:cc}, we learned that \cognicryptsast{} is field-sensitive. These results may be due to the fact that the \cryptoapibench{} test cases are biased in favor of \cryptoguard{} and are designed only to examine whether \cryptoguard{} is field-sensitive or not and therefore do not provide completely field-sensitive cases or are insufficient to analyze field sensitivity. It may also be due to \cognicryptsast{} being only partly field-sensitive. In order to ensure that, it is necessary to test more field-sensitive test cases or to verify that the field-sensitive test cases in \cryptoapibench{} address the field-sensitivity issue. Due to our limited time, we suggest performing this as a future work discussed in Chapter \ref{ch:fwork}.


\cognicryptsast{} detected all of the inter-procedural misuses, which makes it inter-procedural. \codyze{} found 10 of the 11 misuses in the inter-procedural test cases, making it partially inter-procedural. The fact that \codyze{} is inter-procedural is unexpected. As stated in section \ref{sec:codyze}, based on the \codyze{} Java docs and the fact that CPG is not inter-procedural, we concluded that \codyze{} is not inter-procedural. The contradiction may be because the \codyze{} Java documentation has not been updated yet, and \codyze{} is, in fact, inter-procedural, or it is the test cases problem. It is possible that the tests are not inter-procedural, or they are insufficient to determine if an analysis is inter-procedural. Our future work could include verifying that the inter-procedural tests of \cryptoapibench{} are indeed inter-procedural, and providing enough inter-procedural test cases to test the tools to determine if they are inter-procedural.

Both tools have the same precision in each of the advanced test cases separately, but overall \codyze{} produces one more false positive than \cognicryptsast{} (in the path-sensitive case). The path-sensitive case consists of 11 test cases that do not contain any misuses, and if the analyzer detects misuses in those 11 tests, then it is not path-sensitive. \codyze{} detected that all the correct test cases have misuses which indicates that \codyze{} is not path-sensitive.
\cognicryptsast{} found misuses in 10 out of 11 correct path-sensitive test cases, suggesting that \cognicryptsast{} is rarely path-sensitive. However, there is a possibility that \cognicryptsast{} did not find a misuse in this particular case by accident, since as mentioned in Section \ref{sec:cc}, \cognicryptsast is not path-sensitive. \cryptoapibench's test cases may not be sufficient in this case. As future work, we could use more path-sensitive test cases to verify whether the tools are path-sensitive or not.
\\

For similar misuses that both \codyze{} and \cognicryptsast{} detect, they produce different error messages. Table \ref{tab:errormsgs} presents the error messages that \codyze{} and \cognicryptsast{} produce for the misuses they found in four mutual threat models. As we mentioned previously, there is a JSON file (findingdescription.json) produced by the \codyze{} developers, which contains all the possible error messages (cf. section \ref{sec:codyze}), and \cognicryptsast{} generates the error messages based on the type of error (e.g., constraints, predicates, etc.), as we explained in the section \ref{sec:cc}. Therefore, \codyze{} always displays similar error messages when the same misuse occurs. As an example, it always generated the same error message for the incorrect Cipher as demonstrated in the Table \ref{tab:errormsgs} (symmetric cipher threat model). \codyze{} error messages are more abstract than \cognicryptsast{} error messages, as \codyze{} states the misuse of the API and the reason for its insecurity. \cognicryptsast{} error messages provide more details. For instance, in the mode of operation, symmetric cipher, and cryptographic hash, \cognicryptsast{} provides information for the appropriate algorithms, modes, and hashes to use, respectively. In the initialization vector, \codyze{} only mentions that the IV has an inadequate quality, whereas \cognicryptsast{} specifies that the IV needs to be a random number.

% \newcolumntype{P}[1]{>{\centering\arraybackslash}p{#1}}
\begin{table}[H]
\centering

\begin{tabular}{*{3}{p{.32\linewidth}}}
% \caption{Long table caption.\label{long}}\\
 \hline
\diagbox{Threat Model}{Tool} & \codyze & \cognicryptsast \\
\hline
\toprule
Mode of Operation & Use of an unspecified cipher mode for symmetric-key algorithms: A cipher mode for symmetric-key algorithms was detected that does not match one of the recommended Cipher modes by BSI TR-02102. Use of weak or unspecified cipher modes may not guarantee sufficient security. & First parameter (with value \emph{"the parameter of cipher.getInstance(String)"}) should be any of the \emph{"{list of possible modes for that algorithm}"} \\\midrule
Initialization Vector& Insufficient quality of IV for CBC cipher mode: The IV used with the cipher mode CBC is not sufficiently unpredictable. Predictable IVs, which an attacker could guess, compromises the security 
 guarantees of CBC cipher mode.& First parameter was not properly generated as randomized\\\midrule
 
Symmetric Cipher& Use of an unspecified cipher: A cipher was detected that does not match one of the recommended ciphers by BSI TR-02102. Use of weak or unspecified ciphers may not guarantee sufficient security.& First parameter (with value \emph{"the parameter of cipher.getInstance(String)"}) should be any of \emph{"list of all valid parameter of cipher.getInstance(String)"}\\\midrule

Cryptographic Hash& Use of an unspecified hash function: An unspecified hash function is being used. Unspecified hash functions may be weak to attacks. & First parameter (with value \emph{"the parameter of MessageDigest.getInstance(String)"}) should be any of \emph{{list of all valid parameters of MessageDigest.getInstance(String)}}\\
\hline

\end{tabular}
\caption{\label{tab:errormsgs} Comparison of the error messages produced by \codyze{} and \cognicryptsast{} for the similar misuses they found after analyzing \cryptoapibench.}
\end{table}


In summary, overall \codyze{} achieved a higher level of precision than \cognicryptsast{} in analyzing \cryptoapibench{}, however \cognicryptsast{} covered more cryptographic vulnerabilities than \codyze. Therefore, when analyzing the test cases for which both tools have rules (Table \ref{tab:cryptobench}), \cognicryptsast's recall is noticeably higher than \codyze. \codyze{} analysis is context- and flow-sensitive, and partially field-sensitive and inter-procedural, but not path-sensitive. \cognicryptsast{} analysis is inter-procedural and context-sensitive and partially field-sensitive, but seldom path-sensitive.

Furthermore, \cognicryptsast{} produces more detailed error messages than \codyze. There are also some issues with both tools. \codyze{}, for instance, produced duplicate errors when the Cipher order was violated. \codyze{} did not generate any false positives other than those identified in analyzing \cryptoapibench{} in Tables \ref{tab:cryptobench} and \ref{tab:cryptobenchadvanced}. \cognicryptsast{} produces a false positive, that is related to the incorrect usage of the keyGenerator API, in several test cases. However, the misuses of the KeyGenerator API are not considered in the \cryptoapibench{} threat models. Therefore, we did not count them as false positives in the evaluation.

\cryptoapibench{} only covered a few cryptographic vulnerabilities, and the number of test cases they provided may not be sufficient for a fair comparison. Furthuremore, \cryptoapibench{} was created to evaluate \cryptoguard's ability to detect misuses in usages of Java cryptographic APIs, and may therefore be biased toward \cryptoguard. Consequently, we will evaluate \cognicryptsast{} and \codyze{} in the next section with the test cases generated by the \cognicrypttestgen.


% To calculate the runtime we analyzed the \cryptoapibench{} with \cogcnicryptsast{} and \codyze{} on the command line and calculated the time each tool took to generate the result of the analysis. to do so we generated a jar file of \cryptoapibench{} project via Eclipse export option. Based on the information provided on the \cryptoapibench{} Github page, we can build a jar file of \cryptoapibench's project. However, we could not build the jar file with that way due to an error. We have reported this issue on GitHub\footnote{https://github.com/CryptoGuardOSS/cryptoapi-bench/issues/4}. For analyzing \cryptoapibench{} with \codyze{} on the command line we just needed the path to the folder containing the test cases.
\label{sec:apibench}

\section{Evaluation with \cognicrypttestgen}
Results of the analysis in the last section may not provide us with a fair comparison since the number of test cases may not be sufficient to provide a meaningful comparison. Additionally, they do not cover all of the vulnerabilities covered by \codyze{} and \cognicryptsast, and the \cryptoapibench{} was designed to evaluate \cryptoguard's cryptographic misuse detection capability. Therefore, we propose to analyze the test cases generated by \cognicryptsast. Here, we discuss how we generated the test cases using \cognicrypttestgen, how we ran the experiment, calculated and recorded misuses, and finally, we discuss the results of the analyses.

With the help of a \cognicrypttestgen{} developer, we set up the source code of the \cognicrypttestgen{} Eclipse plugin and built it within Eclipse (Skype meeting with one of the \cognicrypttestgen{} developers on Nov 15, 2021). We used the unpublished version of \cognicrypttestgen{} on the evaluation branch\footnote{\url{https://github.com/CROSSINGTUD/CogniCrypt_TESTGEN/tree/evaluation}} to the date of October 31\textsubscript{st} 2021 to the commit e834927\footnote{\url{https://github.com/CROSSINGTUD/CogniCrypt_TESTGEN/commit/e8349276c172bae24b6e1016c309fcd2b5b5d761}}.

In order to generate test cases, \cognicrypttestgen{} utilizes \crysl{} rules. \cognicrypttestgen{} covers most sections in the \crysl{} rules completely, partially covers the ENSURED and CONSTRAINTS sections, and does not cover the FORBIDDEN and NEGATES sections \cite{rakshit}. The generated tests could be used as unit or integration tests \cite{rakshit}. We used them as integration tests in our case because several test cases generated by \cognicrypttestgen{} were incorrectly considered valid or invalid or included more than one misuse and therefore needed further examination.

We used \cognicrypttestgen{} to generate test cases covering all the \MARK{} and \crysl{} rules for the Bouncy Castle JCA and JCA APIs. Therefore, we used JCA and Bouncy Castle JCA \crysl{} rules and the translated Bouncy Castle \crysl{} rules that we created in Section \ref{sec:marktocrysl}. \cognicrypttestgen{} generated test cases for 38 Java cryptographic APIs with JCA \crysl{} rules (see Table \ref{appendix:jcatestcomp}), and it generated test cases for 38 Java cryptographic APIs with Bouncy Castle JCA \crysl{} rules (see Table \ref{tab:bctestcomp}). \cognicrypttestgen{} also generated test cases for 32 Java cryptographic APIs with translated Bouncy Castle \crysl{} rules. There are several valid and invalid test cases for each of those APIs. \cognicrypttestgen{} creates a valid test case for every complete sequence in the FSM (finite state machine) defined in the ORDER section with all the possible methods and parameters and creates an invalid test case for every incomplete sequence in the FSM. Therefore, in invalid test cases, the insecurity is the incorrect order of operators in the use of an API.

To simplify, we will refer to the test cases that \cognicrypttestgen{} generated from \crysl{} Bouncy Castle JCA ruleset as the Bouncy Castle tests and the test cases generated from the JCA \crysl{} ruleset as the JCA tests. Moreover, we refer to the test cases generated from the translated Bouncy Castle \crysl{} rules as the \MARK{} Bouncy Castle tests. In total, there are 228 valid and 233 invalid test cases in the JCA tests and 228 valid and 233 invalid test cases in the Bouncy Castle tests. Furthermore, there are 50 valid and 0 invalid test cases in the Bouncy Castle \MARK{} tests.


We encountered some errors when generating test cases with Bouncy Castle JCA \crysl{} rules, and the \cognicrypttestgen{} could not generate the test cases. We debugged the \cognicrypttestgen's source code and fixed it with the assistance of one of the \cognicrypttestgen's developers (Skype meeting with one of the \cognicrypttestgen{} developers on Nov 26, 2021). We provided a pull request to solve this issue\footnote{\url{https://github.com/CROSSINGTUD/CogniCrypt\_TESTGEN/pull/15}}.



\subsection{Description of the execution of the experiment}
We analyzed the test cases using both tools on the virtual machines as we described in the previous section (see Section \ref{sec:apibench}). Analyzing the test cases, we found that some of the valid test cases contained misuses, while some of the invalid test cases contained no misuses. The misuses in the valid test cases and the correct invalid test cases were detected by both tools. Some of them were detected by \cognicryptsast, and others by \codyze. We will tackle all misuses further when discussing the results. We checked the remaining test cases in which none of the tools detected API misuse manually to ensure there was no API misuse. We considered all the misuses that each tool found in the test cases (both valid and invalid) in the results of the analyses. Moreover, the total number of misuses (for calculating precision and recall) was calculated by measuring the union of the set of misuses found by both tools. We will discuss the calculation of true and false positives in detail in Sections \ref{sec:testgenbcresult} and \ref{sec:testgenmarkbcresult}.
We conducted eight analyses, four for each tool, as follows:
\begin{itemize}
  \item Analyzing Bouncy Castle tests using \codyze{} with Bouncy Castle \MARK{} rules and using \cognicryptsast{} with the translated Bouncy Castle \crysl{} rules.
  
  \item Analyzing Bouncy Castle tests using \codyze{} with translated Bouncy Castle JCA \MARK{} rules and using \cognicryptsast{} with Bouncy Castle JCA \crysl{} rules.
  
  \item Analyzing JCA tests using \codyze{} with translated JCA \MARK{} rules and using \cognicryptsast{} with JCA \crysl{} rules.
  
  \item Analyzing \MARK{} Bouncy Castle tests using \codyze{} with the Bouncy Castle \MARK{} rules and using \cognicryptsast{} with translated Bouncy Castle \crysl{} rules.

\end{itemize}
The true positives and false positives are calculated the same as in the previous section (see Section \ref{sec:apibench}).

\subsection{Examining the results of analyses}
Here, the results of all analyses are compared and discussed.
The \MARK{} Bouncy Castle tests consisted of 50 valid test cases and no invalid ones, with 9 of those 50 valid tests containing a compiler error (see Table \ref{appendix:markbctestcomp}). Consequently, when analyzing these test cases using \codyze{} with the Bouncy Castle \MARK{} rules and \cognicryptsast{} with the translated Bouncy Castle \crysl{} rules, the analysis results as shown in the Table \ref{appendix:markbctestcomp} in the Appendices indicate no misuses, except for the IvParameterSpec test cases. Two of the nine compiler errors occur in the two test cases for the IvParameterSpec API. Despite the compiler errors, \codyze{} detected two misuses in the IvParameterSpec tests, which were a result of the unspecified provider name of the SecureRandom API, which is a true positive (Listing \ref{lst:ivparammarktestgen} line \ref{line:secrandprovider}). \cognicryptsast{} did not detect any misuses in all the cases. Nevertheless, when we removed the erroneous lines from the IvParameterSpec test cases, including the Line \ref{line:faultyline} from Listing \ref{lst:ivparammarktestgen}, \cognicryptsast{} detected the provider misuses and produced two other errors that were related to this misuse. For example, on Line \ref{line:secrandprovider} of listing \ref{lst:ivparammarktestgen} the related error stated that the \code{secureRandom0} parameter was not appropriately generated as a random number, which is also a true positive. However, it is the same as provider misuse. We will discuss the dependent misuses more in Section \ref{sec:testgenbcresult}. Even though both analyses detected the same problem after the fix, this finding illustrates the capability of \codyze{} to analyze erroneous test cases. Since this test case does not provide any additional useful results, we will not discuss it further.

\begin{lstlisting}[language=Java, caption=A faulty test case from \MARK{} Bouncy Castle test, label={lst:ivparammarktestgen}, escapechar=|]
	public void ivParameterSpecValidTest2() throws NoSuchAlgorithmException, NoSuchProviderException {

		String algorithm = null;
		byte[] seed = null;
		int next = 0;
		String provider = null;
		SecureRandom secureRandom0 = SecureRandom.getInstance(algorithm, provider);|\label{line:secrandprovider}|
		secureRandom0.setSeed(seed);
		secureRandom0.next(next); |\label{line:faultyline}|
		int offset = 0;
		int len = 0;
		IvParameterSpec ivParameterSpec0 = new IvParameterSpec(seed, offset, len);

	}
\end{lstlisting}

In the following, we will examine the results of analyzing Bouncy Castle and JCA tests with \codyze{} and \cognicryptsast. The results of analyzing Bouncy Castle tests with Bouncy Castle \crysl{} rules (original and the translation into \MARK), and those of analyzing JCA tests with the JCA \crysl{} rules (original and the translation into \MARK) are similar. Therefore, we focus our discussion on only one of them but provide results for both in tables. Since the Bounce Castle tests provide more information and cover the results of analyses on JCA tests, we will discuss the results of analyzing the Bouncy Castle tests in greater detail. We can therefore divide the analyses results into two sections: first, the results of analyzing Bouncy Castle and JCA tests using \codyze{} and translated Bouncy Castle JCA and JCA \MARK{} rules, and using \cognicryptsast{} and Bouncy Castle JCA and JCA \crysl{} rules, and second, the results of analyzing Bouncy Castle tests using \codyze{} and Bouncy Castle \MARK{} rules, and using \cognicryptsast{} and translated Bouncy Castle \crysl{} rules.


\subsubsection{Discussing the results of analyzing Bouncy Castle and JCA tests using \codyze{} with translated Bouncy Castle JCA and JCA \MARK{} rules, and using \cognicryptsast{} with Bouncy Castle JCA and JCA \crysl{} rules}
\label{sec:testgenbcresult}

In this section, we discuss only the results of analyzing the Bouncy Castle tests generated by \cognicrypttestgen. Because the results of analyzing the two test cases were similar, we chose Bouncy Castle tests since they covered more misuses and contained the misuses in the JCA tests' results. 

Table \ref{tab:bctestcomp} presents the results of analyzing Bouncy Castle tests with \codyze{} and \cognicryptsast. Test name refers to the name of the API for which \cognicrypttestgen{} generated test cases. TP refers to the number of true positives. An error is a true positive when it is caused by a violation of a \crysl{} rule or its translation to \MARK. FP is the number of false positives. An error is a false positive when no violation of the corresponding \crysl{} rule or its translation into \MARK{} has occurred.


Valid tests and invalid tests columns indicate the number of valid and invalid test cases in each of the 38 test cases, according to the \cognicrypttestgen{} division of valid and invalid test cases. In total, there are 228 valid test cases and 233 invalid test cases in the Bouncy Castle tests. However, \codyze{} and \cognicryptsast{} discovered misuses in both valid and invalid test cases. The test cases generated by \cognicrypttestgen{} contained several misuses that were not intended to exist. Some test cases contained more than one misuse, therefore in some cases in the table, the number of true positives is greater than the number of test cases. For example, for the AlgorithmParameters API, there are nine valid test cases, and \codyze{} and \cognicryptsast{} both detected 18 misuses in all nine valid test cases. Therefore, we analyzed misuses that \codyze{} and \cognicryptsast{} detected based on \crysl{} rules and calculated the number of false positives and true positives.


\newcolumntype{P}[1]{>{\centering\arraybackslash}p{#1}}
\newcommand{\mytoprule}{\specialrule{0.1em}{0em}{0em}}
\newcommand{\mybottomrule}{\specialrule{0.1em}{0em}{0em}}
\begin{table}[H]
%  \begin{adjustwidth}{-9em}{} 
\centering
\setlength\tabcolsep{4.31pt}
\small
\begin{tabularx}{\textwidth}{|P{0.5cm}|l|x|x|x|x|}
 \hline
 No.&Test Name & \makecell{Valid \\ tests} & \makecell{Invalid \\ tests} & \begin{tabular}{cccc}
    \multicolumn{4}{c}{\codyze{}} \\
    \multicolumn{2}{c}{Valid} & \multicolumn{2}{c}{Invalid}
    \\TP&FP& TP & FP
\end{tabular} & \begin{tabular}{cccc}
    \multicolumn{4}{c}{\cognicryptsast{}}\\
    \multicolumn{2}{c}{Valid} & \multicolumn{2}{c}{Invalid}\\
    TP&FP&TP&FP

\end{tabular}\\

 \hline
 \hline
1&AlgorithmParameterGenerator&  5 & 9 & 3{\hskip 0.25in}2 {\hskip 0.15in} 12 {\hskip 0.2in} 2&3 {\hskip 0.2in}0 {\hskip 0.15in} 12 {\hskip 0.15in} 0\\
2&AlgorithmParameters& 9& 5& 9{\hskip 0.25in}11 {\hskip 0.15in} 5 {\hskip 0.2in} 5&18 {\hskip 0.15in}0 {\hskip 0.2in} 5 {\hskip 0.2in} 0{\hskip 0.25in}\\
3&CertPathTrustManagerParameters& 1& 0& 0{\hskip 0.25in}0 {\hskip 0.2in} 0 {\hskip 0.2in} 0&0 {\hskip 0.2in}0 {\hskip 0.2in} 0 {\hskip 0.2in} 0\\
4&CipherInputStream& 3& 5& 3{\hskip 0.25in}0 {\hskip 0.2in} 5 {\hskip 0.2in} 0&1 {\hskip 0.2in}0 {\hskip 0.2in} 6 {\hskip 0.2in} 0\\
5&CipherOutputStream& 3&5 &3{\hskip 0.25in}0 {\hskip 0.2in} 5 {\hskip 0.2in} 0&1 {\hskip 0.2in}0 {\hskip 0.2in} 6 {\hskip 0.2in} 0 \\
6&Cipher& 56&69 & 40{\hskip 0.2in}21 {\hskip 0.1in} 72 {\hskip 0.15in} 0{\hskip 0.2in}&40 {\hskip 0.15in}13 {\hskip 0.15in} 90 {\hskip 0.1in} 25{\hskip 0.05in}\\
7&DHGenParameterSpec& 1&0 &1{\hskip 0.25in}0 {\hskip 0.2in} 0 {\hskip 0.2in} 0&1 {\hskip 0.2in}0 {\hskip 0.2in} 0 {\hskip 0.2in} 0\\
8&DHParameterSpec&2&0 &0{\hskip 0.25in}0 {\hskip 0.2in} 0 {\hskip 0.2in} 0&0 {\hskip 0.2in}0 {\hskip 0.2in} 0 {\hskip 0.2in} 0\\
9&DigestInputStream&2&4 &0{\hskip 0.25in}0 {\hskip 0.2in} 0 {\hskip 0.2in} 0&1 {\hskip 0.2in}2 {\hskip 0.2in} 5 {\hskip 0.2in} 0\\
10&DigestOutputStream&2&4 &0{\hskip 0.25in}0 {\hskip 0.2in} 0 {\hskip 0.2in} 0&1 {\hskip 0.2in}2 {\hskip 0.2in} 5 {\hskip 0.2in} 0\\
11&DSAGenParameterSpec&2&0 &0{\hskip 0.25in}0 {\hskip 0.2in} 0 {\hskip 0.2in} 0&0 {\hskip 0.2in}0 {\hskip 0.2in} 0 {\hskip 0.2in} 0\\
12&DSAParameterSpec&1&0 &0{\hskip 0.25in}0 {\hskip 0.2in} 0 {\hskip 0.2in} 0&0 {\hskip 0.2in}0 {\hskip 0.2in} 0 {\hskip 0.2in} 0 \\
13&ECGenParameterSpec&1&0 &0{\hskip 0.25in}0 {\hskip 0.2in} 0 {\hskip 0.2in} 0&0 {\hskip 0.2in}0 {\hskip 0.2in} 0 {\hskip 0.2in} 0 \\
14&ECParameterSpec&1&0 &0{\hskip 0.25in}0 {\hskip 0.2in} 0 {\hskip 0.2in} 0&0 {\hskip 0.2in}0 {\hskip 0.2in} 0 {\hskip 0.2in} 0\\
15&GCMParameterSpec&2&0 &0{\hskip 0.25in}2 {\hskip 0.2in} 0 {\hskip 0.2in} 0&1 {\hskip 0.2in}0 {\hskip 0.2in} 0 {\hskip 0.2in} 0 \\
16&HMACParameterSpec&1&0 &0{\hskip 0.25in}0 {\hskip 0.2in} 0 {\hskip 0.2in} 0&0 {\hskip 0.2in}0 {\hskip 0.2in} 0 {\hskip 0.2in} 0 \\
17&IvParameterSpec&2&0 &0{\hskip 0.25in}0 {\hskip 0.2in} 0 {\hskip 0.2in} 0&1 {\hskip 0.2in}0 {\hskip 0.2in} 0 {\hskip 0.2in} 0 \\
18&KeyAgreement&7&23 &0{\hskip 0.25in}2 {\hskip 0.15in} 23 {\hskip 0.2in} 6&0 {\hskip 0.2in}0 {\hskip 0.15in} 23 {\hskip 0.2in} 0 \\
19&KeyFactory&6&0 &0{\hskip 0.25in}4 {\hskip 0.2in} 0 {\hskip 0.2in} 0&0 {\hskip 0.2in}0 {\hskip 0.2in} 0 {\hskip 0.2in} 0\\
20&KeyGenerator&8&8 &0{\hskip 0.25in}3 {\hskip 0.2in} 8 {\hskip 0.2in} 3&0 {\hskip 0.2in}0 {\hskip 0.2in} 8 {\hskip 0.2in} 0\\
21&KeyManagerFactory&6&4 &0{\hskip 0.25in}8 {\hskip 0.2in} 4 {\hskip 0.2in} 0&0 {\hskip 0.2in}0 {\hskip 0.2in} 4 {\hskip 0.2in} 0\\
22&KeyPairGenerator&6&10 &4{\hskip 0.25in}0 {\hskip 0.15in} 14 {\hskip 0.2in} 0&4 {\hskip 0.2in}0 {\hskip 0.15in} 12 {\hskip 0.2in} 0\\
23&KeyStoreBuilderParameters&1&0 &0{\hskip 0.25in}0 {\hskip 0.2in} 0 {\hskip 0.2in} 0&0 {\hskip 0.2in}0 {\hskip 0.2in} 0 {\hskip 0.2in} 0\\
24&KeyStore&13&21&0{\hskip 0.25in}0 {\hskip 0.15in} 17 {\hskip 0.2in} 0&0 {\hskip 0.2in}0 {\hskip 0.15in} 17 {\hskip 0.2in} 0\\
25&Mac&12&21 &0{\hskip 0.25in}7 {\hskip 0.15in} 21 {\hskip 0.2in} 0&1 {\hskip 0.2in}0 {\hskip 0.15in} 22 {\hskip 0.2in} 0\\
26&MessageDigest&9&13 &0{\hskip 0.25in}1 {\hskip 0.2in} 2 {\hskip 0.2in} 1&2 {\hskip 0.2in}0 {\hskip 0.2in} 8 {\hskip 0.2in} 0\\
27&MGF1ParameterSpec&1&0 &0{\hskip 0.25in}0 {\hskip 0.2in} 0 {\hskip 0.2in} 0&0 {\hskip 0.2in}0 {\hskip 0.2in} 0 {\hskip 0.2in} 0\\
28&OAEPParameterSpec&1&0 &0{\hskip 0.25in}0 {\hskip 0.2in} 0 {\hskip 0.2in} 0&0 {\hskip 0.2in}0 {\hskip 0.2in} 0 {\hskip 0.2in} 0\\
29&PBEKeySpec&1&1 &0{\hskip 0.25in}1 {\hskip 0.2in} 1 {\hskip 0.2in} 1&0 {\hskip 0.2in}0 {\hskip 0.2in} 1 {\hskip 0.2in} 0\\
30&PBEParameterSpec&2&0 &0{\hskip 0.25in}1 {\hskip 0.2in} 0 {\hskip 0.2in} 0&0 {\hskip 0.2in}0 {\hskip 0.2in} 0 {\hskip 0.2in} 0\\
31&PKIXBuilderParameters&2&0 &0{\hskip 0.25in}0 {\hskip 0.2in} 0 {\hskip 0.2in} 0&0 {\hskip 0.2in}0 {\hskip 0.2in} 0 {\hskip 0.2in} 0\\
32&PKIXParameters&2&0 &0{\hskip 0.25in}0 {\hskip 0.2in} 0 {\hskip 0.2in} 0&0 {\hskip 0.2in}0 {\hskip 0.2in} 0 {\hskip 0.2in} 0\\
33&RSAKeyGenParameterSpec&1&0 &0{\hskip 0.25in}0 {\hskip 0.2in} 0 {\hskip 0.2in} 0&0 {\hskip 0.2in}0 {\hskip 0.2in} 0 {\hskip 0.2in} 0\\
34&SecretKeyFactory&3&2 &0{\hskip 0.25in}4 {\hskip 0.2in} 2 {\hskip 0.2in} 0&0 {\hskip 0.2in}0 {\hskip 0.2in} 2 {\hskip 0.2in} 0\\
35&SecureRandom&38&16 &3{\hskip 0.25in}0 {\hskip 0.2in} 3 {\hskip 0.2in} 0&0 {\hskip 0.2in}2 {\hskip 0.2in} 0 {\hskip 0.2in} 7\\
36&SSLContext&5&5&10{\hskip 0.2in}0 {\hskip 0.2in} 5 {\hskip 0.2in} 0&10 {\hskip 0.15in}0 {\hskip 0.2in} 5 {\hskip 0.2in} 0\\
37&SSLParameters&4&4 &0{\hskip 0.25in}1 {\hskip 0.2in} 4 {\hskip 0.2in} 0&0 {\hskip 0.2in}0 {\hskip 0.2in} 3 {\hskip 0.2in} 0\\
38&TrustManagerFactory&6&4 &0{\hskip 0.2in}13 {\hskip 0.2in} 4 {\hskip 0.2in} 0&0 {\hskip 0.2in}0 {\hskip 0.2in} 4 {\hskip 0.2in} 0\\
 \hline
 \hline
 \multicolumn{2}{|l|}{ \textbf{Total}} &  \textbf{228} & \textbf{233} & 76 {\hskip 0.1in} 81{\hskip 0.2in} 207 {\hskip 0.1in}18 &  85{\hskip 0.1in} 19{\hskip 0.2in} 238{\hskip 0.1in} 32\\
 \hline
 \multicolumn{2}{|l|}{\multirow{2}{*}{\textbf{Results}}}
 &   \multicolumn{2}{|l|}{\textbf{Precision(\%)}} &74,08 & 86,36\\ 
 
 \multicolumn{2}{|l|}{} & \multicolumn{2}{|l|}{\textbf{Recall(\%)}} & 85,75 & 97,87 \\
 \hline

\end{tabularx}
\caption{\label{tab:bctestcomp} Comparison of the results of \codyze{} and \cognicryptsast{} analyses of Bouncy Castle tests generated by \cognicrypttestgen.}
% \end{adjustwidth}
\end{table}

% in most of the cases the amount of true positives are similar except for the cipher where there is a big difference between codyze findings and cognicryptsast findings.
\cognicryptsast{} generates misuses that depend upon other misuses.
For instance, both tools have found misuses in the following code in listing \ref{lst:ciphertestgen}. \cognicryptsast{} detects four misuses that are all true positives in this test case, and \codyze{} finds three misuses that are also true positives.
The tools have two common misuses. First, on line \ref{line:wrongsize}, the parameter of AlgorithmParameterGenerator's \code{init} method must be in any of 128, 192, 256 bits and not 1048 bits. Second, the Cipher algorithm should not be RSA. In addition, \cognicryptsast{} identifies two more misuses on Line \ref{line:wrongorder} that are caused by the first two misuses. The error indicates that the second parameter (\code{secretKey}) was not properly generated as a generated key, which is correct since the algorithms used in Cipher and KeyGenerator are different. The other error states that the third parameter (\code{algorithmParameters}) was not generated properly, which is also true due to the incorrect parameter used in the AlgorithmParameterGenerator. The dependent errors will be resolved by correcting the first two misuses. We have found more dependent errors in other cipher test cases but none in other test cases. \codyze{} detects another misuse that \cognicryptsast{} does not, which is the incorrect order of Cipher in Line \ref{line:wrongorder}. This misuse is caused by the fact that the Cipher has not been terminated properly, and a final call is required.


We counted the true positives for Cipher without the dependent errors in the Table \ref{tab:bctestcomp}. If we count the dependent misuses, then \cognicryptsast{} finds 71 true positives in the valid Cipher tests and 104 true positives in the invalid tests, which contain many duplicate errors. Therefore, we did not include the dependent errors when calculating precision and recall.
% as it appears in the table \ref{tab:bctestcomp}, both tools produce several false positives in the Cipher test cases. most of these false positives relate to the order misuses. 

Suppose we only count the misuses caused by incorrect order of operations, as shown in the Table \ref{appendix:bctestcomporder} in the appendices. In that case, we can see that \codyze{} performed better in finding order misuses than \cognicryptsast{} in the Cipher test cases. If we calculate the precision and recall just for the Cipher test case for all valid and invalid cases, \codyze{} achieves 72,72\% precision and 100\% recall, and \cognicryptsast{} achieves 67,14 \% precision and 83,92 \% recall, which indicates that in this case, \codyze{} performs better than \cognicryptsast. The overall precisions of \cognicryptsast{} and \codyze{} were very similar; however, \codyze's recall was about 3 percent higher, indicating that it was more effective in finding order misuses.


\begin{lstlisting}[language=Java, caption=One of the valid test cases for the Cipher API that \cognicrypttestgen{} generated from the original \crysl{} Bouncy Castle JCA ruleset., label={lst:ciphertestgen}, escapechar=|]
	public void cipherValidTest7() throws NoSuchPaddingException, NoSuchAlgorithmException, InvalidKeyException, InvalidAlgorithmParameterException {

		KeyGenerator keyGenerator0 = KeyGenerator.getInstance("AES");
		SecretKey secretKey = keyGenerator0.generateKey();

		AlgorithmParameterGenerator algorithmParameterGenerator0 =                   AlgorithmParameterGenerator.getInstance("AES");
		algorithmParameterGenerator0.init(1048);|\label{line:wrongsize}|
		AlgorithmParameters algorithmParameters = algorithmParameterGenerator0.generateParameters();

		Cipher cipher0 = Cipher.getInstance("RSA"); |\label{line:wrongalg}|
		cipher0.init(1, secretKey, algorithmParameters); |\label{line:wrongorder}|
	}
\end{lstlisting} 

\cognicryptsast{} detects some misuses that \codyze{} does not, and therefore the number of true positives for \cognicryptsast{} in some cases are more than \codyze. Those misuses are related to the constraints error. As an example, in the Cipher \crysl{} rule in the CONSTRAINTS section, it is stated that the input length of the Cipher object must be greater than 0. In the Listing \ref{lst:partciphtestgen} Line \ref{line:conscipher}, the third parameter, which is the input length, is zero; therefore it is a misuse, which \cognicryptsast{} identified. However, \codyze{} does not detect this misuse, even though it has a relative \MARK{} rule.

\begin{lstlisting}[language=Java, caption=Part of a test case for the Cipher API from the Bouncy Castle tests, label={lst:partciphtestgen}, escapechar=|]
...
		Cipher cipher0 = Cipher.getInstance("RSA");
		cipher0.updateAAD(aadBytes);
		cipher0.doFinal(plainText, 0, 0, cipherText, 0); |\label{line:conscipher}|
...
\end{lstlisting}

In some cases, \codyze{} reports that a rule was violated and verified at the same time where there was no misuse, and we counted the violation as a false positive. For example, the listing \ref{lst:algorithmtestgen} shows one of the invalid test cases of AlgorithmParameterGenerator from Bouncy Castle tests. \codyze{} identifies a violation of a rule on Line \ref{line:notrandom} that states that the parameter \code{secureRandom0}, that was used in AlgorithmParameterGenerator was not generated by the SecureRandom API. This error is a false positive since the SecureRandom API was correctly used to generate a random number (\code{secureRandom0}). On the same line (Line \ref{line:notrandom}), \codyze{} identifies a rule verification that states that the \code{secureRandom0} was generated correctly with the SecureRandom API. This is in conflict with the previous misuse. This unusual behavior was also observed in several other test cases.
There is also the same violation and verification at Line \ref{line:repeatorder}, which is redundant. On Line \ref{line:repeatorder} of Listing \ref{lst:algorithmtestgen}, \codyze{} detected a violation against order because AlgorithmParameterGenerator was not properly terminated. It is a true positive, but in the Markers tab of Eclipse, this error had been displayed several times, which is unnecessary.



\begin{lstlisting}[language=Java, caption=One of the invalid test cases for the AlgorithmParameterGenerator API that \cognicrypttestgen{} generated from the original \crysl{} JCA ruleset., label={lst:algorithmtestgen}, escapechar=|]
	public void algorithmParameterGeneratorInvalidTest5() throws NoSuchAlgorithmException {

		SecureRandom secureRandom0 = SecureRandom.getInstance("DEFAULT");|\label{line:notrandom}|

		AlgorithmParameterGenerator algorithmParameterGenerator0 = AlgorithmParameterGenerator.getInstance("AES");
		algorithmParameterGenerator0.init(1048, secureRandom0); |\label{line:repeatorder}|

	}
\end{lstlisting}


Four \MARK{} policies did not function, and as a result, they did not report any violations or verifications. They were \MARK{} policies for the following APIs: DigestInputStream and DigestOutputStream, as well as CipherInputStream and CipherOutputStream. We checked for syntax issues in the relative files (entity and rule files), but there were no problems, and all of these policies were written the same as others and based on the instructions on \codyze's documentation page \cite{cod}. \codyze, however, detected misuses in two of those API's related test cases (i.e., numbers 4 and 5 of Table \ref{tab:bctestcomp}) that were related to other APIs. In the CipherInputStream and CipherOutputStream tests, \codyze{} found violations in the order of Cipher in all of the test cases (valid and invalid), which are true positives. It was the same error that appeared in Listing \ref{lst:ciphertestgen} Line \ref{line:wrongorder}, which indicated that the Cipher was not correctly terminated.

The results of analyzing JCA tests using \codyze{} with translated JCA \MARK{} rules and using \cognicryptast{} with JCA \crysl{} rules are shown in the Table \ref{appendix:jcatestcomp} in the appendices. There were fewer incorrect test cases in the JCA tests than in the Bouncy Castle tests. Therefore, the number of misuses that each tool detected in total was less in the JCA tests than in the Bouncy Castle tests. As a result, there were fewer dependent errors in the Cipher test cases in the JCA tests than in the Bouncy Castle tests. The number of misuses in Table \ref{appendix:jcatestcomp} is calculated without considering the dependant errors. If we count the dependent errors, the number of misuses \cognicryptast{} found in the valid Cipher cases will be 57 and in invalid 101. \codyze{} could not find some constraints errors in the JCA tests, as it did in the Bouncy Castle tests. Because the misuses found by each tool in the JCA tests are similar to those found in the Bouncy Castle tests, we will not discuss them again.


\subsubsection{Discussing the results of analyzing Bouncy Castle tests using \codyze{} with Bouncy Castle \MARK{} rules, and \cognicryptsast{} with the translated Bouncy Castle \crysl{} rules.}
\label{sec:testgenmarkbcresult}
We further analyzed the Bouncy Castle tests using \codyze{} with the Bouncy Castle \MARK{} rules and \cognicryptsast{} with translated Bouncy Castle \MARK{} rules. The results of the analyses are presented in Table \ref{tab:bctestcompmarkrules}. In Table \ref{tab:bctestcompmarkrules}, all column titles are the same as those in Table \ref{tab:bctestcomp} discussed in section \ref{sec:testgenbcresult}. TP refers to the number of true positives. An error is a true positive when it is caused by a violation of a \MARK{} rule or its translation to \crysl. FP is the number of false positives. An error is a false positive when no violation of the corresponding \MARK{} rule or its translation into \crysl{} has occurred.

In total, \codyze{} identified three types of misuses in the test - invalid or non-specified provider name, non- or incorrect padding for RSA Cipher, using the forbidden calls of SecureRandom API, and one invalid size of the key in the RSAKeyGenParameterSpec test case, all of which were true positives. \cognicryptsast{} identified all of those types of misuses but in fewer cases. \cognicryptsast{} generally found fewer misuses than \codyze. For example, in the Cipher tests (number 6 in Table \ref{tab:bctestcompmarkrules}), it appears that \cognicryptsast{} did not identify as many misuses as \codyze. This is because \cognicryptsast{} was unable to detect all of the provider misuses. The reason is the partial translation of the \MARK{} rules to \crysl. Not indicating a provider when using an API is considered a misuse, as stated in section \ref{sec:marktocrysl}. To translate this rule to \crysl, we must add the methods without a provider parameter to the FORBIDDEN section of the \crysl{} rule, which was not possible for all methods within an API. Therefore \cognicryptsast{} could not detect all the provider misuses. The same holds for SecureRandom forbidden calls. Since we could not add all the forbidden methods (see Listing \ref{lst:orgsecureRandomMARK} Lines \ref{line:forbidpbe1} and \ref{line:forbidpbe2}) in the FORBIDDEN section of SecureRandom \crysl{} rule; therefore, \cognicryptsast{} could not detect all forbidden calls.


As indicated in the Table \ref{tab:bctestcompmarkrules}, \codyze{} does not generate any false positives, while \cognicryptsast{} generates some false positives. In some tests, \cognicryptsast{} detects order misuses, which are all false positives. The order specified in the rules includes only optional usage of the methods, so any order is acceptable. Therefore, any order misuse generated by \cognicryptsast{} is a false positive. As an example, the false positives in the Mac tests in the valid and invalid test cases generated by \cognicryptsast{} are all caused by order misuses. \cognicryptsast generates order errors in Cipher, SecureRandom, Mac, and KeyAgreement test cases that are all false positives as described. An invalid order error can only be generated when using the Cipher API, with the Cipher mode being one of CCM, GCM, CBC, or CTR. This particular case requires specific method calls according to the \MARK{} Bouncy Castle rules. However, the Cipher mode was not specified in any of the Bouncy Castle test cases.


\cognicryptsast{} did not detect all the non- or incorrect padding for RSA Cipher misuses. Based on the Bouncy Castle \MARK{} rules, if an RSA Cipher is used without padding or with incorrect padding in the \code{getInstance} method of Cipher, this error should occur. This rule is defined as a constraint in the Cipher \crysl{} rule. \cognicryptsast{} can only find misuses of the parameters used within the EVENTS section. According to the \MARK{} rules, the \code{getInstance} method without a provider was considered insecure. However, this method could not be added to the FORBIDDEN section of the Cipher \crysl{} rule, as explained in Section \ref{sec:marktocrysl}. Therefore, \cognicryptsast{} does not detect RSA padding misuse.
Moreover, the other misuse of the RSAKeyGenParameterSpec API, regarding the incorrect size of the key, occurred in only one test case and was detected by both tools.



\begin{table}[H]
\centering
\setlength\tabcolsep{4.3pt} 
\small
\begin{tabularx}{\textwidth}{|P{0.5cm}|l|x|x|x|x|}
 \hline
 No.&Test Name & \makecell{Valid \\ tests} & \makecell{Invalid \\ tests} & \begin{tabular}{cccc}
    \multicolumn{4}{c}{\codyze{}} \\
    \multicolumn{2}{c}{Valid} & \multicolumn{2}{c}{Invalid}
    \\TP&FP& TP & FP
\end{tabular} & \begin{tabular}{cccc}
    \multicolumn{4}{c}{\cognicryptsast{}}\\
    \multicolumn{2}{c}{Valid} & \multicolumn{2}{c}{Invalid}\\
    TP&FP&TP&FP

\end{tabular}\\

 \hline
 \hline
1&AlgorithmParameterGenerator&  5 & 9 & 7{\hskip 0.25in}0 {\hskip 0.15in} 11 {\hskip 0.15in} 0&1 {\hskip 0.2in}0 {\hskip 0.2in} 3 {\hskip 0.2in} 0\\
2&AlgorithmParameters& 9& 5& 13{\hskip 0.2in}0 {\hskip 0.15in} 5 {\hskip 0.2in} 0&2 {\hskip 0.2in}0 {\hskip 0.2in} 2 {\hskip 0.2in} 0\\
3&CertPathTrustManagerParameters& 1& 0& 1{\hskip 0.25in}0 {\hskip 0.15in} 0 {\hskip 0.2in} 0&0 {\hskip 0.2in}0 {\hskip 0.2in} 0 {\hskip 0.2in} 0\\
4&CipherInputStream& 3& 5& 6{\hskip 0.25in}0 {\hskip 0.15in} 10 {\hskip 0.15in} 0&0 {\hskip 0.2in}0 {\hskip 0.2in} 0 {\hskip 0.2in} 0\\
5&CipherOutputStream& 3&5 &6{\hskip 0.25in}0 {\hskip 0.15in} 10 {\hskip 0.15in} 0&0 {\hskip 0.2in}0 {\hskip 0.2in} 0 {\hskip 0.2in} 0 \\
6&Cipher& 56&69 & 164{\hskip 0.12in}0 {\hskip 0.12in} 177 {\hskip 0.1in} 0&8 {\hskip 0.2in}1 {\hskip 0.15in} 21 {\hskip 0.2in} 0\\
7&DHGenParameterSpec& 1&0 &0{\hskip 0.25in}0 {\hskip 0.2in} 0 {\hskip 0.2in} 0&0 {\hskip 0.2in}0 {\hskip 0.2in} 0 {\hskip 0.2in} 0\\
8&DHParameterSpec&2&0 &0{\hskip 0.25in}0 {\hskip 0.2in} 0 {\hskip 0.2in} 0&0 {\hskip 0.2in}0 {\hskip 0.2in} 0 {\hskip 0.2in} 0\\
9&DigestInputStream&2&4 &2{\hskip 0.25in}0 {\hskip 0.2in} 4 {\hskip 0.2in} 0&0 {\hskip 0.2in}0 {\hskip 0.2in} 0 {\hskip 0.2in} 0\\
10&DigestOutputStream&2&4 &2{\hskip 0.25in}0 {\hskip 0.2in} 4 {\hskip 0.2in} 0&0 {\hskip 0.2in}0 {\hskip 0.2in} 0 {\hskip 0.2in} 0\\
11&DSAGenParameterSpec&2&0 &0{\hskip 0.25in}0 {\hskip 0.2in} 0 {\hskip 0.2in} 0&0 {\hskip 0.2in}0 {\hskip 0.2in} 0 {\hskip 0.2in} 0\\
12&DSAParameterSpec&1&0 &0{\hskip 0.25in}0 {\hskip 0.2in} 0 {\hskip 0.2in} 0&0 {\hskip 0.2in}0 {\hskip 0.2in} 0 {\hskip 0.2in} 0 \\
13&ECGenParameterSpec&1&0 &0{\hskip 0.25in}0 {\hskip 0.2in} 0 {\hskip 0.2in} 0&0 {\hskip 0.2in}0 {\hskip 0.2in} 0 {\hskip 0.2in} 0 \\
14&ECParameterSpec&1&0 &0{\hskip 0.25in}0 {\hskip 0.2in} 0 {\hskip 0.2in} 0&0 {\hskip 0.2in}0 {\hskip 0.2in} 0 {\hskip 0.2in} 0\\
15&GCMParameterSpec&2&0 &2{\hskip 0.25in}0 {\hskip 0.2in} 0 {\hskip 0.2in} 0&0 {\hskip 0.2in}0 {\hskip 0.2in} 0 {\hskip 0.2in} 0 \\
16&HMACParameterSpec&1&0 &0{\hskip 0.25in}0 {\hskip 0.2in} 0 {\hskip 0.2in} 0&0 {\hskip 0.2in}0 {\hskip 0.2in} 0 {\hskip 0.2in} 0 \\
17&IvParameterSpec&2&0 &2{\hskip 0.25in}0 {\hskip 0.2in} 0 {\hskip 0.2in} 0&0 {\hskip 0.2in}2 {\hskip 0.2in} 0 {\hskip 0.2in} 0 \\
18&KeyAgreement&7&23 &10{\hskip 0.2in}0 {\hskip 0.15in} 28 {\hskip 0.15in} 0&1 {\hskip 0.2in}0 {\hskip 0.2in} 5 {\hskip 0.2in} 1 \\
19&KeyFactory&6&0 &10{\hskip 0.2in}0 {\hskip 0.2in} 0 {\hskip 0.2in} 0&3 {\hskip 0.2in}0 {\hskip 0.2in} 0 {\hskip 0.2in} 0\\
20&KeyGenerator&8&8 &11{\hskip 0.2in}0 {\hskip 0.15in} 11 {\hskip 0.15in} 0&2 {\hskip 0.2in}0 {\hskip 0.2in} 2 {\hskip 0.2in} 0\\
21&KeyManagerFactory&6&4 &6{\hskip 0.25in}0 {\hskip 0.15in} 0 {\hskip 0.2in} 0&0 {\hskip 0.2in}0 {\hskip 0.2in} 0 {\hskip 0.2in} 0\\
22&KeyPairGenerator&6&10 &6{\hskip 0.25in}0 {\hskip 0.15in} 10 {\hskip 0.15in} 0&1 {\hskip 0.2in}0 {\hskip 0.2in} 3 {\hskip 0.2in} 0\\
23&KeyStoreBuilderParameters&1&0 &0{\hskip 0.25in}0 {\hskip 0.15in} 0 {\hskip 0.2in} 0&0 {\hskip 0.2in}0 {\hskip 0.2in} 0 {\hskip 0.2in} 0\\
24&KeyStore&13&21&13{\hskip 0.2in}0 {\hskip 0.17in} 21 {\hskip 0.15in} 0&4 {\hskip 0.2in}0 {\hskip 0.2in} 6 {\hskip 0.2in} 0\\
25&Mac&12&21 &27{\hskip 0.2in}0 {\hskip 0.15in} 35 {\hskip 0.15in} 0&0 {\hskip 0.2in}1 {\hskip 0.2in} 0 {\hskip 0.2in} 2\\
26&MessageDigest&9&13 &9{\hskip 0.25in}0 {\hskip 0.15in} 13 {\hskip 0.15in} 0&2 {\hskip 0.2in}0 {\hskip 0.2in} 3 {\hskip 0.2in} 0\\
27&MGF1ParameterSpec&1&0 &0{\hskip 0.25in}0 {\hskip 0.2in} 0 {\hskip 0.2in} 0&0 {\hskip 0.2in}0 {\hskip 0.2in} 0 {\hskip 0.2in} 0\\
28&OAEPParameterSpec&1&0 &0{\hskip 0.25in}0 {\hskip 0.2in} 0 {\hskip 0.2in} 0&0 {\hskip 0.2in}0 {\hskip 0.2in} 0 {\hskip 0.2in} 0\\
29&PBEKeySpec&1&1 &1{\hskip 0.25in}0 {\hskip 0.2in} 1 {\hskip 0.2in} 0&0 {\hskip 0.2in}0 {\hskip 0.2in} 0 {\hskip 0.2in} 0\\
30&PBEParameterSpec&2&0 &2{\hskip 0.25in}0 {\hskip 0.2in} 0 {\hskip 0.2in} 0&0 {\hskip 0.2in}0 {\hskip 0.2in} 0 {\hskip 0.2in} 0\\
31&PKIXBuilderParameters&2&0 &1{\hskip 0.25in}0 {\hskip 0.2in} 0 {\hskip 0.2in} 0&0 {\hskip 0.2in}0 {\hskip 0.2in} 0 {\hskip 0.2in} 0\\
32&PKIXParameters&2&0 &1{\hskip 0.25in}0 {\hskip 0.2in} 0 {\hskip 0.2in} 0&0 {\hskip 0.2in}0 {\hskip 0.2in} 0 {\hskip 0.2in} 0\\
33&RSAKeyGenParameterSpec&1&0 &1{\hskip 0.25in}0 {\hskip 0.2in} 0 {\hskip 0.2in} 0&1 {\hskip 0.2in}0 {\hskip 0.2in} 0 {\hskip 0.2in} 0\\
34&SecretKeyFactory&3&2 &5{\hskip 0.25in}0 {\hskip 0.2in} 2 {\hskip 0.2in} 0&1 {\hskip 0.2in}1 {\hskip 0.2in} 1 {\hskip 0.2in} 0\\
35&SecureRandom&38&16 &71{\hskip 0.17in}0 {\hskip 0.15in} 19 {\hskip 0.15in} 0&10 {\hskip 0.15in}1 {\hskip 0.2in} 4 {\hskip 0.2in} 2\\
36&SSLContext&5&5&5{\hskip 0.25in}0 {\hskip 0.2in} 0 {\hskip 0.2in} 0&0 {\hskip 0.2in}0 {\hskip 0.2in} 0 {\hskip 0.2in} 0\\
37&SSLParameters&4&4 &0{\hskip 0.25in}0 {\hskip 0.2in} 0 {\hskip 0.2in} 0&0 {\hskip 0.2in}0 {\hskip 0.2in} 0 {\hskip 0.2in} 0\\
38&TrustManagerFactory&6&4 &6{\hskip 0.25in}0 {\hskip 0.2in} 0 {\hskip 0.2in} 0&0 {\hskip 0.2in}0 {\hskip 0.2in} 0 {\hskip 0.2in} 0\\
 \hline
 \hline
 \multicolumn{2}{|l|}{ \textbf{Total}} &  \textbf{228} & \textbf{233} & 390 {\hskip 0.05in} 0{\hskip 0.2in} 361 {\hskip 0.15in}0 &  36{\hskip 0.15in} 6{\hskip 0.2in} 50{\hskip 0.2in} 5\\
 \hline
  \multicolumn{2}{|l|}{\multirow{2}{*}{\textbf{Results}}}
 &   \multicolumn{2}{|l|}{\textbf{Precision(\%)}} &100 & 88,65\\ 
 
 \multicolumn{2}{|l|}{} & \multicolumn{2}{|l|}{\textbf{Recall(\%)}} & 100 & 11,45 \\
 \hline

\end{tabularx}
\caption{\label{tab:bctestcompmarkrules} Comparison of the results of \codyze{} and \cognicryptsast{} analyses with \MARK{} rules of Bouncy Castle tests generated by \cognicrypttestgen.}
\end{table}


To summarize, \cognicryptsast{} provided better precision and recall than \codyze{} in analyzing the Bouncy Castle and JCA tests; however, both tools performed well in general. Nevertheless, when comparing errors relating to order misuses, \codyze{} performs better than \cognicryptsast.

The \cognicryptsast{} with translated \crysl{} rules did not perform well in identifying the misuses based on \MARK{} rules. According to Table \ref{tab:bctestcompmarkrules}, \cognicryptsast{} only found 11 percent of the misuses. This could indicate that \MARK{} is a more expressive DSL than \crysl. \cognicrypttestgen{} generated tests, however, did not cover all sections of the \crysl{} rules. To determine whether all translated \MARK{} rules deliver the same performance as the \crysl{} rules, we need a benchmark that covers all vulnerabilities related to all parts of the \crysl{} rules.

In addition, the rules defined by \MARK{} in the Bouncy Castle ruleset did not address many of the cryptographic vulnerabilities we know today, such as the use of non-random keys in encryption. The available \crysl{} rules, in contrast, cover many known vulnerabilities. By analyzing the test cases generated by the \crysl{} rules using translated \MARK{} rules in \codyze, we were able to understand how \codyze{} would perform with the new rules that define more vulnerabilities. 


Although the test cases generated by \cognicrypttestgen{} did not cover all sections of the \crysl{} rules, they still contain many vulnerabilities that detecting them could be very useful. Notably, the test cases included parts of the \crysl{} rules that could be translated to mark. Thus, we expected \codyze{} to be similar to \cognicryptsast{} in terms of rules. However, after the analysis, we discovered that some rules behave strangely, and some constraint rules do not work. Additionally, a few \MARK{} policies did not function at all. The results also showed that \cognicryptsast{} could not detect all order misuses and, in some cases, find order misuses that are false positives.


\label{sec:testgen}