As of the date of this thesis, there have been no papers published on \codyze{}, and the documentation page is brief \cite{cod}. Therefore, to understand \codyze{}, we investigated its source code on Github \cite{codyzegit}, its documentation page and consulted with one of \codyze's developers (personal communications with one of the \codyze's developers via several emails and an online meeting on Sep 17, 2021).

\codyze{} \cite{cod} is an open-source static analysis tool for Java, C and C++ programs developed by Fraunhofer AISEC\footnote{https://www.aisec.fraunhofer.de/}. It generates a code property graph (CPG) \cite{cpg} from the source code and analyzes it against predefined \MARK{} \cite{cod} rules for finding cryptographic misuses. \MARK{} is a domain-specific language (DSL) designed to write correct usages of Java, C, and C++ cryptographic APIs. Code property graphs allow \codyze{} to handle non-compiling or incomplete code. A CPG is a directed multigraph, which means that two nodes may be connected by multiple edges. Listing \ref{lst:sampleCCPG} illustrates an example code that shows a source method leaking (potentially) sensitive data and a sink method that may expose the data. Figure \ref{fig:CPG} depicts a CPG representation of Listing \ref{lst:sampleCCPG}, in which there are multiple edges between the nodes PRED and DECL. The nodes represent program constructs. For example, in figure \ref{fig:CPG}, PRED is a predicate node that indicates the condition of the if statement in Listing \ref{lst:sampleCCPG}, Line \ref{line:PRED} and DECL refers to the declaration statements of Line \ref{line:DECL} in the Listing \ref{lst:sampleCCPG}.
This graph incorporates the properties of Abstract Syntax Tree (AST), Control Flow Graph (CFG), and Program Dependence Graph (PDG) in a single data structure \cite{cpg}. An AST represents the abstract syntactic structure of the source code. It is an ordered tree with inner nodes representing operators and leaf nodes representing operands \cite{cpg}. A CFG is a graph that represents all the possible paths that can be traversed during program execution. It is a directed graph that explicitly describes the order in which code statements are executed as well as conditions that need to be met for a particular path of execution to be taken \cite{controlflowgraph}. The program dependence graph indicates explicit dependencies between statements and predicates. Vertices in PDG represent the statements and predicates in the program. The edges representing the dependencies between components fall into two categories; data dependency edges reflect the influence of one variable upon another, and control dependency edges reflect the influence of predicates on the values of variables. After constructing each graph individually, the three graphs are merged to form the CPG. Graph nodes are the same as AST nodes, edges and labels are combinations of all three graphs, and sets of property keys and property values for nodes and edges are from AST and PDG graphs. Property keys and values as well as labels on AST edges are not shown in Figure \ref{fig:CPG}. CFG and PDG edges are indicated by colors.


\begin{lstlisting}[caption= Code example for making it into a CPG \cite{cpg}. , label={lst:sampleCCPG}, backgroundcolor =  , xleftmargin=.3\textwidth, frame = single, xrightmargin=.3\textwidth, escapechar=|]
void foo() 
 {  
    int x = source(); 
    if (x < MAX) |\label{line:PRED}|
    { 
        int y = 2 * x; |\label{line:DECL}|
        sink(y); 
    } 
}

\end{lstlisting}
\begin{figure}[H]
\centering
\includegraphics[width=1\linewidth]{figures/cpg-original.png}
\caption{Code property graph for code sample in Listing \ref{lst:sampleCCPG} \cite{cpg}.}
\label{fig:CPG}
\end{figure}

Observing \codyze's source code, we discovered that \codyze{} is a flow-sensitive static data-flow analysis. Through the use of CPGs and graph traversals, \codyze{} is able to access the code flow and data dependencies associated with each node, thereby providing a data-flow analysis and can check the nodes in the CPG (e.g., methods, variables, etc.) against the defined \MARK{} rules. In addition, there are \MARK{} rules that specify the correct order of using an API's methods\footnote{\url{https://github.com/Fraunhofer-AISEC/codyze/blob/0468af19b90b16353402938db4e326b6dc63c4f7/src/dist/mark/bouncycastle/RulesBase_Cipher.mark}}, which indicates that \codyze{} is flow-sensitive. CPG and the query language are not enough to determine whether the program complies with the \MARK{} order rules, so \codyze{} developers implemented an intra-procedural typestate analysis. By using typestate analysis, \codyze{} determines whether a sequence of operations is valid based on the \MARK{} rules\footnote{\url{https://github.com/Fraunhofer-AISEC/codyze/blob/e67d5fa0a1c956ce4be53fcdecf1990eb8dfb430/src/main/java/de/fraunhofer/aisec/codyze/analysis/wpds/TypestateWeight.java}}.
CPG does not cover inter-procedural analysis \cite{cpg}. Further, it is stated in the Java documentation for \codyze's implementation of typestate analysis that it is not inter-procedural; Thus, \codyze{} is not considered to be inter-procedural. Nevertheless, when evaluating \codyze's performance, we discovered that \codyze{} is partially inter-procedural. At the time of this thesis, no published papers about \codyze{} were available; therefore, we were unable to identify any other properties of it. However, we can better assess the analysis properties of \codyze{} through evaluation of its analysis of example codes. We will discuss it further in the evaluation chapter (cf. Chapter \ref{ch:eval}).


\codyze{} runs in three modes, interactive command line, non-interactive command line, and as a server for Language Server Protocol (LSP) \cite{lsp}. \codyze{} uses the LSP in order to provide greater flexibility in the use of different IDEs and supporting several languages. The LSP describes the communication protocol used between an editor or IDE and a language server, which enables features such as auto-complete, go to definition, etc. With a standard communication protocol, a single language server can be re-used in various code editors without too much effort \cite{lsp}. \codyze{} creates a language server that converts the source code from a client (IDE or command line) to a CPG. \codyze{} developers have implemented a parser that parses \MARK{} rules into a model that the analysis server can use. The analysis server analyzes the CPG against the \MARK{} model with the help of Crymlin queries \cite{gremlin}. Crymlin is an extension of the Apache Gremlin graph traversal language that provides a variety of shortcuts for searching through code property graphs \cite{gremlin}. 


\codyze{} also provides a JSON file (findingDescription.json) that contains all the possible errors based on the \MARK{} rules in a human-readable format and is stored in the same location as the \MARK{} rules (Listing \ref{lst:findings}). This file may later be used to create error messages in the client. It contains all possible errors, with the exception of errors related to the incorrect order of method calls and the use of insecure methods. The error messages for these two violations are generated automatically at runtime, since the error message for the first violation (the incorrect order) contains the missing methods\footnote{\url{https://github.com/Fraunhofer-AISEC/codyze/blob/0468af19b90b16353402938db4e326b6dc63c4f7/src/main/java/de/fraunhofer/aisec/codyze/analysis/markevaluation/OrderNFAEvaluator.java}} and the message for the second violation contains the forbidden call\footnote{\url{https://github.com/Fraunhofer-AISEC/codyze/blob/04b5db6029a1bc6f8ad680cfecc796f408705031/src/main/java/de/fraunhofer/aisec/codyze/analysis/markevaluation/ForbiddenEvaluator.kt}}.

Listing \ref{lst:findings} shows a sample error description in the findingDescription JSON file generated by \codyze{}. The error refers to the incorrect cryptographic algorithm or the absence of an algorithm during the creation of the Cipher object. The first line (Line \ref{line:errorname}) is the name of the error. HelpUri (Line \ref{line:helpuri}) contains a URL from which \MARK{} rules are derived (Bundesamt für Sicherheit in der Informationstechnik\footnote{Federal Office for Information Security} or BSI \cite{BSI}). FullDescription (Line \ref{line:fulldesc}) provides a complete description of the error, while shortDescription (Line \ref{line:shortdesc}) is only a concise description. Line \ref{line:fixes} describes the possible solutions to the problem. The fix, in this case, is to use the AES or RSA algorithm.
\pagebreak
\begin{lstlisting}[language=json, caption= {Part of findingDescription.json file generated by \codyze{} after analyzing Listing \ref{lst:codesample}}, label={lst:findings}, escapechar=@]
...
{
    ...
    "Invalid_TR21021_Cipher": { @\label{line:errorname}@
    "helpUri": "https://www.bsi.bund.de/SharedDocs/Downloads/DE/BSI/Publikationen@\label{line:helpuri}@
    /TechnischeRichtlinien/TR02102/BSI-TR-02102.pdf",
    "fullDescription": {@\label{line:fulldesc}@
      "text": "A cipher was detected that does not match one of the recommended ciphers by BSI TR-02102. Use of weak or unspecified ciphers may not guarantee sufficient security."
    },
    "shortDescription": {@\label{line:shortdesc}@
      "text": "Use of an unspecified cipher"
    },
    "fixes": [@\label{line:fixes}@
      {
        "description": {
          "text": "Use AES as symmetric-key (cf. 2.1) or RSA for asymmetric-key algorithm (cf. 3)"
        }}]
    }
    ...
 }
 ...
\end{lstlisting}


As well as spotting cryptographic misuses in a program, \codyze{} will also alert if a \MARK{} rule is being used correctly. \codyze{} can be used as a console application or integrated into IDEs or CI pipelines. In the time of this thesis, \codyze{} is integrated into Eclipse, Visual Studio, and IntelliJ \cite{cod}. \codyze{} offers a standard Python console for exploring the source code project using the Crymlin query interface.

