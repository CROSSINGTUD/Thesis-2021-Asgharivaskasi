\cognicryptsast{} \cite{stefanphd} is an open-source static analyzer that analyses Java codes to find misuses of cryptographic APIs. \cognicryptsast{} is integrated into the \cognicrypt{} Eclipse plugin and can also be used as a standalone command-line tool to analyze Android and Java applications \cite{cryptoanalysis}. \cognicryptsast{} consumes \crysl{} \cite{skm19} rules to configure a flow-sensitive and context-sensitive static data-flow analysis. Flow-sensitivity is the ability to determine the order in which statements are presented \cite{pointer}. The context-sensitive analysis is an inter-procedural analysis that considers the calling context when analyzing a method \cite{pointer}. An intra-procedural analysis analyzes a single procedure, while an inter-procedural analysis examines the interrelationships among procedures. Data flow analysis refers to the process of tracking data through a program in order to explain its behavior without actually executing the program \cite{johannesphd}. According to Krueger \cite{stefanphd}, \cognicryptsast{} is also field-sensitive but not path-sensitive. Path-sensitive data flow analysis improves the precision of data flow analysis by analyzing the feasibility of paths \cite{pathsensitive}.

\cognicryptsast{} uses the Soot framework \cite{soot} for static analysis. Soot enables Java developers to build their own static analyzer tool. It contains several intra-procedural and inter-procedural features to solve intra- and inter-procedural analysis problems. Direct analysis of Java bytecode can be hindered by the unclarity of data flow; hence \cognicryptsast{} employs Soot that offers several intermediate representations (IR) of Java code. Jimple is the fundamental Soot IR that simplifies analyzing and making a graph. \cognicryptsast{} builds a Call Graph (CG) from Jimple code. A call graph is a directed graph that represents the calling relation in a program function. Nodes represent methods, and edges represent a call relation between two methods \cite{callgraph}. In addition, analysis employs CFGs to identify the control flow of each method. For performing inter-procedural analysis of a complete program, the \cognicryptsast's analysis combines individual control flow graphs and the program's call graph in order to construct an inter-procedural control flow graph (ICFG).

\cognicryptsast{} utilizes CryptoAnalysis to build CGs and perform the analysis. CryptoAnalysis is a compiler that translates rules to static analysis. CryptoAnalysis warns the developer if any part of the specified \crysl{} rules is violated in the code. Three different static sub-analyses form CryptoAnalysis: typestate analysis, a Boomerang \cite{boomerang} instant and taint analysis for Android applications. Typestate analysis checks the valid sequence of operations based on the defined order on \crysl{} rules. For instance, the call sequence of the KeyGenerator object is \code{getInstance}, \code{init} (optional), and then \code{generatekey} (see Listing \ref{lst:orgkeygencrysl} Line \ref{line:ordercrysl}). Any other combinations will be false. CryptoAnalysis uses Boomerang, a demand-driven pointer analysis for Java, to extract parameters on the fly to check them against the defined \crysl{} rules. For example, it checks if a valid \code{algorithm} is chosen for the \code{getInstance} method of KeyGenerator.
Cryptoanalysis also includes taint analysis, which detects flaw injections or leaks of sensitive information. For analyzing Android applications, CryptoAnalysis employs Flowdroid \cite{flowdroid} which is a static taint analysis for Android applications.

It is possible to select different kinds of CG algorithms from \cognicryptsast, CHA (Class Hierarchy Analysis), SPARK \cite{spark} and SPARK\_LIBRARY that are available through command-line commands or Eclipse preference pages. The CHA is the default algorithm, which is not precise but is efficient. SPARK is a framework for call graph construction and points-to analysis, and it is a part of the Soot framework. SPARK's call graph construction algorithm considers any call that may occur at any point in the execution of the program \cite{soot}. SPARK implements a context-insensitive points-to analysis. The points-to analysis associates variables with the allocation sites \cite{pointer}. An analysis that is context-insensitive merges all call sites of a procedure.
SPARK\_LIBRARY is a SPARK mode specifically designed for the analysis of libraries. For all classes within the library, dummy allocation sites are instantiated, resulting in non-empty points-to sets for variables within the library \cite{johannesphd}. 

Misuses in \cognicryptsast{} are categorised into 7 categories, namely, ConstraintError, NeverTypeOfError, ForbiddenMethodError, ImpreciseValueExtractionError, TypestateError, RequiredPredicateError, IncompleteOperationError \cite{cryptoanalysis}. The error messages that \cognicryptsast{} generates depend on the type of misuse and will be automatically generated at the end of the analysis.