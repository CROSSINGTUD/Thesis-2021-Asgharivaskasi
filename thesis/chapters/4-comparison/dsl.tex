\crysl{} \cite{skm19} and \MARK{} \cite{cod} are the DSLs used by \cognicryptsast{} \cite{skm19} and \codyze{} \cite{cod}, respectively, to specify the correct usages of cryptographic APIs. 
This section aims to compare the DSLs on a theoretical and practical level fairly. Therefore, we performed a systematic review \cite{systematic4} of ways to compare DSLs. We searched on different platforms like Google\footnote{https://www.google.com}, GoogleScholar\footnote{https://scholar.google.com}, IEEE explore digital library\footnote{https://ieeexplore.ieee.org}, and ResearchGate\footnote{https://www.researchgate.net} for related articles and read the relevant ones, but there is no publication that compares DSLs in the same domain in terms of expressiveness. Rather, we found similar yet unrelated papers. For example, Cuadrado \etal{} compared two DSLs with the same syntax but different implementation techniques (internal and external) in their paper \cite{evaldsl14}. But that is not suitable for our DSLs since \MARK{} and \crysl{} are both external DSLs and this comparison is not useful. The other papers discussed comparing DSLs and GPLs (General Purpose Languages) to test program comprehension via user studies, which is not the focus of our study \cite{dslvsgpl12} or measuring usage simplicity in DSLs and GPLs \cite{empdslvsgpl10}, or comparisons of ways to define DSLs \cite{comptext8}, which are all unrelated to our research. Several studies tested usability \cite{usabil12} \cite{usabledsl11} \cite{qualitydsl11} \cite{successfac09}, which requires a user study and is beyond the scope of this thesis.

Thus, we propose translating \crysl{} rules to \MARK{} rules, and vice versa, to explore their expressiveness.
By translating one language to another and vice versa in a particular domain, such as defining correct usages of cryptographic APIs, we are expressing elements of one language in another while preserving the meaning, which demonstrates the expressiveness in that particular domain. There will be four possible scenarios in our case of translation. If we can translate \MARK{} rules into \crysl{} and \crysl{} rules into \MARK, then \MARK{} and \crysl{} are equally expressive when defining rules for cryptographic APIs.	However, if \crysl{} rules can be expressed by \MARK, but \MARK{} rules cannot be translated into \crysl, then \MARK{} is more expressive in specifying cryptographic API rules. Additionally, if \MARK{} rules could be expressed by \crysl{}, and \crysl{} rules could not be expressed by \MARK{}, then \crysl{} is more expressive than \MARK{} in defining rules for cryptographic APIs. Finally, if \crysl{} rules and \MARK{} rules cannot be translated into each other, then they express different things in defining rules for cryptographic APIs. 


Prior to comparing the DSLs, we provide a short explanation of \MARK{} and \crysl. We will then compare them practically and translate the rules, and finally, we will compare them theoretically.

\subsection{\crysl{}}
\label{subsec:crysl}
\crysl{} is a domain-specific language (DSL) that allows cryptography experts to describe secure usage of their crypto APIs in a lightweight special-purpose syntax \cite{skm19}. The \crySL compiler is built upon Xtext \cite{xtext}, an open-source framework for developing domain-specific languages. Based on the \crysl{} grammar, Xtext provides parsing, type checking, and syntax highlighting capabilities. A \crysl{} rule is a simple text file and can be written in any text editor. Eclipse can be used to create \crysl{} rules with syntax correction by installing the \crysl{} editor. 

Currently, \crysl{} rules have been developed by crypto experts of \cognicrypt{} developers for JCA, Bouncy Castle \cite{bc}, a Java library that extends Java Cryptographic Extension (JCE)\footnote{JCE is an API that provides a framework that allows Java developers to implement security features \cite{jce}.}, Bouncy Castle JCA \cite{bc}, Bouncy Castle provider for the JCA, and Tink \cite{tk}, an open-source easy to use and secure cryptographic library made by Google to reduce API misuses \cite{apirules}.

A \crysl{} rule is written for each class and consists of 6 mandatory and 3 optional parts. We introduce the semantics of \crysl{} rules by using the \crysl{} rule for javax.crypto.KeyGenerator in Listing \ref{lst:orgkeygencrysl} as an example. We further use part of the \crysl{} rule for javax.crypto.spec.PBEKeySpec (Listing \ref{lst:orgpbecryslRule}) to elaborate some infrequently used terms.
\pagebreak
\begin{lstlisting}[language=crysl,caption= {KeyGenerator \crysl{} rule from JCA API \cite{apirules}}., label={lst:orgkeygencrysl}, escapechar=^]
SPEC javax.crypto.KeyGenerator ^\label{line:spec}^

OBJECTS
	int keysize; ^\label{line:objects1}^
	java.security.spec.AlgorithmParameterSpec params;
	javax.crypto.SecretKey key;
	java.lang.String algorithm;
	java.security.SecureRandom random;^\label{line:objects2}^

EVENTS
	g1: getInstance(algorithm);^\label{line:g1}^
	g2: getInstance(algorithm, _); ^\label{line:g2}^
	Get := g1 | g2; ^\label{line:get}^

	i1: init(keysize); ^\label{line:i1}^
	i2: init(keysize, random);
	i3: init(params);
	i4: init(params, random);
	i5: init(random); ^\label{line:i2}^
	Init := i1 | i2 | i3 | i4 | i5;
    
	gk1: key = generateKey(); ^\label{line:gk1}^
	GenKey := gk1; ^\label{line:genkey}^

ORDER
	Get, Init?, GenKey ^\label{line:ordercrysl}^

CONSTRAINTS
	algorithm in {"AES", "HmacSHA256", "HmacSHA384", "HmacSHA512"};^\label{line:const1}^
	algorithm in {"AES"} => keysize in {128, 192, 256};^\label{line:const2}^
   
REQUIRES
	randomized[random]; ^\label{line:requirerandom}^
    
ENSURES 
	generatedKey[key, algorithm]; ^\label{line:generatedkey}^
\end{lstlisting}


\subsubsection{Mandatory sections}
On the first line (\ref{line:spec}), there is SPEC that indicates the class name. The OBJECTS (Line \ref{line:objects1} to \ref{line:objects1}) are parameters or return values of methods in the EVENTS section. All methods that may contribute to the successful use of an object of the \crysl{} rule are described in the EVENTS section. Each method pattern is represented by a label in the EVENTS section (e.g., \code{Get} in Line \ref{line:get}). When different patterns of the same method are possible, only one of them based on the implementation will be chosen in the ORDER section. For example, in Listing \ref{lst:orgkeygencrysl} Lines \ref{line:g1} and \ref{line:g2} show two patterns of \code{getInstance} method: \code{g1} with only one parameter, and \code{g2} with two parameters. \crysl{} uses aggregates to represent the disjunction of labels (e.g., \code{Get} in line \ref{line:get}).

The ORDER section defines a usage pattern in the form of regular expression for methods in the EVENTS section. An example would be the \code{getInstance} call (Line~\ref{line:ordercrysl}) followed by the \code{init} call (which is optional), followed by a call to the \code{generateKey} method in Listing \ref{lst:orgkeygencrysl}.

The CONSTRAINTS show constraints for objects. For instance, Line~\ref{line:const1} maintains that \code{algorithm} must be one of the following: AES, HmacSHA256, HmacSHA384, HmacSHA512. The Line \ref{line:const2} indicates that if the \code{algorithm} is AES, then the \code{keySize} must be one of 128, 192, or 256.
 
Assuming that the object is used appropriately, meaning that all limitations in the CONSTRAINTS section and usage patterns in the ORDER section are considered, the ENSURES section defines what a class predicates. Line \ref{line:generatedkey} of the keyGenerator \crysl rule demonstrates that using this class properly ensures the generation of a \code{key} with a specific \code{algorithm}.
  
\crysl{} allows us to define a method-event pattern using the \emph{after} keyword (Line~\ref{line:enskey}). For example, in Line~\ref{line:enskey} of Listing \ref{lst:orgpbecryslRule}, \code{speccedKey} predicate is created if the appropriate method is called. Thus, the PBEKeySpec class ensures the generation of a key after a call to the appropriate constructor with a specified \code{keyLength}.

 

\begin{lstlisting}[language= CrySL,caption= Part of \crysl{} rule for javax.crypto.spec.PBEKeySpec from JCA ruleset \cite{apirules}., label={lst:orgpbecryslRule}, escapechar=|]
SPEC javax.crypto.spec.PBEKeySpec

OBJECTS
    ...
FORBIDDEN
	PBEKeySpec(char[]) => Con; |\label{line:forbidden1}|
	PBEKeySpec(char[],byte[],int) => Con; |\label{line:forbidden2}|
	
EVENTS
	Con: PBEKeySpec(password, salt, iterationCount, keyLength); |\label{line:eventcon}|
	
	ClearPass: clearPassword();|\label{line:clearpass}|
...
CONSTRAINTS
	iterationCount >= 10000;
	neverTypeOf[password, java.lang.String]; |\label{line:nevertyp}|
	notHardCoded[password]; |\label{line:nothardcod}|
...
ENSURES
	speccedKey[this, keyLength] after Con; |\label{line:enskey}|
	
NEGATES
	speccedKey[this, _] after ClearPass; |\label{line:neg}|
\end{lstlisting}


\subsubsection{Optional sections}
Apart from mandatory sections that each \crysl{} rule must have, some \crysl{} rules may contain optional sections. The FORBIDDEN section involves calls to the methods that are insecure. The PBEkeySpec constructor call must consume a salt to generate a more secure key, whereas the signatures of the constructors in Lines~\ref{line:forbidden1} and~\ref{line:forbidden2} do not consume a salt; therefore, those calls are insecure and forbidden. After the \code{clearPassword} call, a PBEKeySpec object made by a constructor is no longer valid. \crysl{} allows invalidating a current predicate in the NEGATES field. As shown in Line~\ref{line:neg}, after \code{clearPassword} is called, the PBEkeySpec object is no longer effective. The predicates that one rule ENSURES can be used in the REQUIRES section of another rule. For example, Line~\ref{line:requirerandom} requires that the salt be generated from a random seed.


\crysl{} developers have added some simple built-in auxiliary functions to make it more expressive. For example, in Listing \ref{lst:orgpbecryslRule}, the function
\code{neverTypeOf} in Line \ref{line:nevertyp} states that the password should never be of type \code{String} and Line \ref{line:nothardcod} states that the password should not be hard-coded. \cite{stefanphd} fully described all of the built-ins in \crysl.
\subsection{\MARK{}}
\label{subsec:mark}
\MARK{} is a domain-specific language for writing rules to specify the correct usage patterns of APIs or libraries \cite{cod}. It stands for "Modellierungssprache fuer Anforderungen und Richtlinien der Kryptografie" (Modeling Language for Cryptography Requirements and Guidelines). \MARK{}'s grammar is written using Xtext. \MARK{}'s Eclipse plugin project \cite{markgithub} provides several features for \MARK{} language, including a language server for using \MARK{} in IDEs with LSP support, an Xtext generator that creates Crymlin/Gremlin from \MARK{} files, and an Eclipse plugin for syntax highlighting, auto-completion, and code correction \cite{markgithub}. Unfortunately, there are no further details available regarding the use of \MARK{} on other IDEs with LSP support or the creation of Crymlin/Gremlin out of \MARK{} files. In order to fully grasp these concepts, it is necessary to conduct extensive research on the implementation of \MARK{}, which is beyond the scope of this thesis.

Currently, there are \MARK{} policies made by \codyze's developers for the Botan \cite{botan}, a C++ cryptography library, Bouncy Castle and Jackson \cite{jackson}, a high-performance JSON processor for Java libraries for cryptography in Java. After reviewing the \MARK{} ruleset on \codyze's Github page, we discovered that the Bouncy Castle \MARK{} ruleset contains rules for the JCA API. However, we will continue to refer to it as the Bouncy Castle \MARK{} ruleset.

Writing \MARK{} rules for a library necessitates a thorough knowledge of the API and class structure of the library. \MARK{} policies are separated into two parts that we will discuss in separate sections; Entities (see Section \ref{subsubsec:entity}) and Rules (see Section \ref{subsubsec:rule}). Rules determine the correct usage of the entities. Entities are an abstract grouping of API functions. Entity contains three parts:
\begin{itemize}
\item \emph{Name}, the name of the rule.
\item \emph{Ops}, set of operations (op).
\item \emph{Var}, variables that refer to function arguments or return values.
\end{itemize}


Here we will provide a brief overview of the entity and rule parts of \MARK{} policies.

\subsubsection{Entity}
\label{subsubsec:entity}
To describe entities, we consider Listing \ref{lst:orgkeygenmarken} which shows the entity part of the \MARK{} policy for javax.crypto.KeyGenerator and Listing \ref{lst:orgsecureRandomMARK} which is a part of \MARK{} entity file for java.security.SecureRandom to describe a less frequently used optional keyword, \code{forbidden}, in \MARK{}.

\begin{lstlisting}[language=MARK,caption= {\MARK{} entities for javax.crypto.KeyGenerator of Bouncy Castle ruleset \cite{codyzegit}}, label={lst:orgkeygenmarken}, escapechar=@]
package java.jca

entity KeyGenerator {
    
    var algorithm;@\label{line:varsbeg}@
    var provider;
    
    var keysize;
    var random;
    var params;
    var key;@\label{line:varsend}@
    
    op instantiate {@\label{line:opsbeg}@
        javax.crypto.KeyGenerator.getInstance(algorithm : java.lang.String); @\label{line:instance1}@
        javax.crypto.KeyGenerator.getInstance(@\label{line:instance2}@
            algorithm : java.lang.String,
            provider : java.lang.String | java.security.Provider
        );
    }
    
    op init {
        javax.crypto.KeyGenerator.init(keysize : int);
        javax.crypto.KeyGenerator.init(
            keysize : int,
            random : java.security.SecureRandom
        );
        javax.crypto.KeyGenerator.init(random : java.security.SecureRandom);
        javax.crypto.KeyGenerator.init(params : java.security.spec.AlgorithmParameterSpecs);
        javax.crypto.KeyGenerator.init(
            params : java.security.spec.AlgorithmParameterSpec,
            random : java.security.SecureRandom
        );
    }
    
    op generate {
        key = javax.crypto.KeyGenerator.generateKey();
    }@\label{line:opsend}@
}
\end{lstlisting}

The first step is to define model relevant classes as \MARK{} entities. Only those classes which contain the relevant data or function must be identified as \MARK{} entities. Even though many programming languages classes can be combined in an abstract \MARK{} entity in several cases, it can at first be easier to map classes explicitly into entities. For instance, KeyGenerator class to entity KeyGenerator. Developers can freely choose the name of the entity \cite{cod}.

The second step is to define ops and variables. An op is a semantically equivalent or related group of functions, methods, or constructors, provided as completely qualified signatures. The most common ops in cryptographic libraries are, instantiate, initialize, update, finalize and reset. For example, in Listing \ref{lst:orgkeygenmarken}, we have instantiate op, which shows the \code{getInstance} methods with different possible parameters (Lines~\ref{line:instance1},~\ref{line:instance2}). The type of each op's parameters with a name is specified, but we can also have unnamed and untyped parameters (described with "\_"). These parameters do not play any role in the definition of rules. Named parameters must also be declared as entity variables using the \code{var} keyword (Lines~\ref{line:varsbeg} to~\ref{line:varsend}) \cite{cod}.

The third step, which is optional, is to blacklist the forbidden ops. In some cases, functions that are deprecated or established to be insecure should not be used in the program. The \code{forbidden} keyword in \MARK{} indicates that the use of the following functions is insecure. For example, Lines~\ref{line:forbidpbe1} and~\ref{line:forbidpbe2} of Listing \ref{lst:orgsecureRandomMARK} are forbidden calls when using the SecureRandom class because they do not respect Bouncy Castle as a provider \cite{cod}.

\begin{lstlisting}[language=MARK,caption= Part of \MARK{} entities for java.security.SecureRandom from Bouncy Castle ruleset \cite{codyzegit}., label={lst:orgsecureRandomMARK}, escapechar=^]
    ...
     op instantiate {
        ...
        java.security.SecureRandom.getInstanceStrong();
        
        // forbidden calls because they don't respect BC as provider
        forbidden java.security.SecureRandom(); ^\label{line:forbidpbe1}^
        forbidden java.security.SecureRandom(^\label{line:forbidpbe2}^
            seed : byte[]
        );
    }
    ...
\end{lstlisting}


\subsubsection{Rule}
\label{subsubsec:rule}
After defining entities, we can start writing rules. \MARK{} rules apply to instances of entities and specify the conditions that must be met in these instances. Instances of \MARK{} may correspond to actual objects but might also be abstract functions or variables in non-object-oriented languages and static methods. Listing \ref{lst:orgkeygenMARKrule} displays one of the \MARK{} rules associated with the KeyGenerator class from the Bouncy Castle \MARK{} files included in \codyze's Github repository.

Every rule has a unique name across all \MARK{} files loaded into \codyze{} (Line \ref{line:rulename}). \code{Using} keyword (Line \ref{line:usingpart}) initiates the declaration of instances of the \MARK{} entities, here KeyGenerator and javax.crypto.Mac instances, and \code{ensure} shows the condition (Line \ref{line:ensurepart}). A violation of the condition will result in a finding with a message indicated by the \code{onfail} identifier (Line \ref{line:onfail}). 


In some rules, some preconditions must be met before the main condition is evaluated. If they fail, the main condition will not be evaluated, and the rule will not return any results. Preconditions are declared by the \code{when} keyword (Line \ref{line:whenpart}). Mark includes several built-in functions that can be used as predicates in conditions and preconditions (e.g., Line \ref{line:builtinmark}). They are called during MARK rule evaluation and operate over their input arguments (usually MARK objects or constants) and the evaluation context. By convention, built-ins should begin with "\_". If a built-in fails, it will return an Error object which evaluates to not applicable, i.e., neither true nor false. 

In the following rule example (Listing \ref{lst:orgkeygenMARKrule}), once the preconditions are satisfied, i.e., if the Mac algorithm is AESCMAC (Line \ref{line:whencond}), and the variable Mac key equals the KeyGenerator key (Line \ref{line:builtinmark}), then the rule condition must be met, that is, the KeyGenerator's key must be greater than or equal to 128 (Line \ref{line:condition2}). Otherwise, the InsufficientCMACKeyLength error will be thrown.

\begin{lstlisting}[language=MARK2,caption= A \MARK{} rule associated with javax.crypto.KeyGenerator class from Bouncy Castle ruleset \cite{codyzegit}., label={lst:orgkeygenMARKrule}, escapechar=^]
rule ID_5_3_02_CMAC_Keygen {  ^\label{line:rulename}^
    using ^\label{line:usingpart}^
        Mac as m,
        KeyGenerator as kg
    when ^\label{line:whenpart}^
        m.algorithm in ["AESCMAC"] ^\label{line:whencond}^
        && _is(m.key, kg.key) ^\label{line:builtinmark}^
    ensure ^\label{line:ensurepart}^
        _is(m.key, kg.key) ^\label{line:condition1}^
        && kg.keysize >= 128 ^\label{line:condition2}^
    onfail  ^\label{line:onfail}^
        InsufficientCMACKeyLength 
}
\end{lstlisting}

In this example, the condition \code{\_is(m.key, kg.key)} is mentioned in both the \code{when} and \code{ensure} sections. It might be a mistake since if it is true in the \code{when} section, then it is also true in the \code{ensure} section, so there is no need to verify it again. The developers of \codyze{} did not elaborate on the matter further; therefore, we have opened an issue on \codyze{}'s Github repository\footnote{https://github.com/Fraunhofer-AISEC/codyze/issues/400}.

 

\subsection{Practical comparison of \crysl{} and \MARK{}}
\label{sec:practicdsl}

This section aims to compare the DSLs on a practical level. In order to accomplish this, we proposed translating \crysl{} rules to \MARK{} and vice versa. Since there are \MARK{} and \crysl{} rules for the Bouncy Castle JCA API for Java, we first translate Bouncy Castle JCA rules from \crysl{} to \MARK{} and vice versa. As stated before, there are no \MARK{} rules specified for the JCA library, which is the primary cryptography API for Java applications \cite{snb16}. In this regard, it is worthwhile to translate \crysl{} JCA rules in \MARK{} language and compare their functionality and result of their analyses on the same Java cryptography examples. So at the end we will have three translations, \crysl{} Bouncy Castle JCA to \MARK, \MARK{} Bouncy Castle JCA to \crysl{} and \crysl{} JCA to \MARK. Source codes for the translated rules are available in Appendices \ref{appendix:ruletranslaions}.

We developed the following translation based on the information we obtained from reviewing existing \MARK{} and \crysl{} rules, \crysl{} paper \cite{skm19} and \codyze's documentation page \cite{cod}. We will employ these translated rules in the evaluation of \codyze{} and \cognicryptsast{} (cf. Chapter \ref{ch:eval}).



\subsubsection{Setup}
\label{sec:rulesetup}
We used the same virtual machines we explained in Section \ref{sec:practical} to translate rules. On the first machine with Java 8, we installed the \crysl{} Eclipse plugin and on the second machine with Java 11, we installed the \MARK{} Eclipse plugin. We used the \crysl{} rules from the Crypto-API-rules Github repository master branch \cite{apirules} to the commit 1dbad34\footnote{\url{https://github.com/CROSSINGTUD/Crypto-API-Rules/commit/1dbad342a46c47df62e891fc25f46944972d9e18}} and \MARK{} rules from \codyze{} Github repository main branch \cite{codyzegit} to the commit 8c74a13\footnote{\url{https://github.com/Fraunhofer-AISEC/codyze/commit/8c74a13be2385b79875990c5e36ed67b39579662}}, to the date of October 31st, 2021.

